%================================================================================
%       Safety Critical Systems Club - Data Safety Initiative Working Group
%================================================================================
%                       DDDD    SSSS  IIIII  W   W   GGGG
%                       D   D  S        I    W   W  G   
%                       D   D   SSS     I    W W W  G  GG
%                       D   D      S    I    WW WW  G   G
%                       DDDD   SSSS   IIIII  W   W   GGG
%================================================================================
%               Data Safety Guidance Document - LaTeX Source File
%================================================================================
%
% Description:
%   Supplier Data Maturity section.
%
%================================================================================
\section{Supplier Data Maturity (Informative)} \label{bkm:maturity}

\dsiwgSectionQuote{The most valuable thing you can have as a leader is clear data.}{Ruth Porat}

This questionnaire may be used for two purposes:
\begin{enumerate}
  \item To support a procurement process - distributed by an organization looking for a company that can handle safety-critical development, because the system they require to be developed is known to have safety-critical requirements
  \item Internal audits - used internally by a company developing systems with safety-related data which needs to assure itself of its capability to fulfil customer needs.
\end{enumerate}

\dsiwgTextBF{Organization}
\begin{enumerate}
  \item For each software development involving data, is there a designated data safety manager?
  \item If so, does the data safety manager report directly to the project manager?
  \item Are the management reporting channels for data assurance and software development separate?
  \item Is data subject to a formal configuration control process? 
  \item Is data engineering represented on the system design team?
  \item Is data engineering process improvement part of the company quality systems?
\end{enumerate}

\dsiwgTextBF{Resources, Personnel and Training}
\begin{enumerate}
  \item Are personnel specified as responsible for data safety as a separate role from software and system design and development?
  \item Is there a required training programme for data specialists?
  \item Is training on data safety issues part of the training for managers or management teams?
  \item Is there a formal training programme for data safety design and review leaders
\end{enumerate}

\dsiwgTextBF{Data Issues Growth Management}
\begin{enumerate}
  \item Is a mechanism employed for maintaining awareness of the state of the art in data safety technology?
  \item Is a mechanism employed for comparing the company approach to data safety with external processes for data safety practised elsewhere in the industry?
  \item Is a mechanism used for introducing new technologies and processes into system development?
  \item Is a mechanism in place for identifying and replacing obsolescent processes related to data safety?
\end{enumerate}

\clearpage% Move heading onto next page (doesn't align with 3.2, but corrects layout error)
\dsiwgTextBF{Documented Standards and Procedures}
\begin{enumerate}
  \item Describe any formal procedures adopted at each periodic management review to assess the status of data related to the system.
  \item Describe the methods used for ensuring that the data development team understands each data requirement.
  \item Is a data risk assessment method used for assessing the use of existing data in new applications?
  \item Are data test cases developed formally with a company standard?
  \item Is there a document which describes how the customer is to be consulted over data issues?
  \item Is particular care taken to capture requirements, design, review and test data for user interfaces\index{Interface!User}?
  \item Is there a data risk monitoring and tracking to closure procedure practised?
\end{enumerate}

\dsiwgTextBF{Process Metrics}
\begin{enumerate}
  \item Are statistics of failures due to data errors during development kept to feedback and learn from in future development?
  \item Are data issue action items tracked to closure and reports maintained of causes?
  \item Is \gls{configuration data} separately developed from everyday operational data?
  \item Is data test coverage measured and recorded?
  \item Are all states, from which \gls{configuration data} will be required, tested, (including emergency reboot), and results recorded?
  \item Are analyses of errors due to data conducted to determine their process related causes?
  \item Are the process causes reviewed and changes to processes implemented where appropriate?
\end{enumerate}

\dsiwgTextBF{Process Control}
\begin{enumerate}
  \item Is regression testing routinely performed when errors are discovered?
  \item Is the adequacy of regression testing subject to an assurance process to ensure new errors are not introduced?
  \item Is a mechanism used for identifying and resolving system engineering issues that affect data?
  \item Is a mechanism used for ensuring traceability between the data requirements and the top-level design?
  \item Is the importance of data in the system engineering process reviewed to maintain processes at an adequate level to cope with the expanding role of data in the Internet of Things? 
\end{enumerate}
