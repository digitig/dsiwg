%================================================================================
%       Safety Critical Systems Club - Data Safety Initiative Working Group
%================================================================================
%                       DDDD    SSSS  IIIII  W   W   GGGG
%                       D   D  S        I    W   W  G   
%                       D   D   SSS     I    W W W  G  GG
%                       D   D      S    I    WW WW  G   G
%                       DDDD   SSSS   IIIII  W   W   GGG
%================================================================================
%               Data Safety Guidance Document - LaTeX Source File
%================================================================================
%
% Description:
%   Activities and Tailoring section.
%
%================================================================================
\section{Activities and Tailoring (Informative)} \label{bkm:activitiestailoring}

\dsiwgSectionQuote
  {It is a capital mistake to theorise before one has data.}
  {Sherlock Holmes - ``A Study in Scarlet'' (Sir Arthur Conan Doyle)}


\subsection{Establish Context}
\subsubsection{Overview}
This phase involves developing: an understanding of the context within which the system development occurs; an understanding of the system requirements; and an understanding of the system design.
These factors help determine the risk appetite; that is, essentially, how much effort will be devoted to making risks as low as practicable. In turn, this will inform the nature and scope of assessments that are conducted during system development, introduction to service and operation. The factors also help identify \index{Stakeholder}Stakeholders.

\subsubsection{Activities}
There are four activities associated with this phase.

\paragraph{Describe the organizational context}
Part of this activity involves understanding the \index{Stakeholder}Stakeholders involved, or with an interest, in the system. It is important to define how the \index{Stakeholder}Stakeholders will interact and the derived requirements applicable to each \index{Stakeholder}Stakeholder through interface control\index{Interface!organizational}, similarly to systems engineering interface control procedures already in place in many industries. Consideration of external (e.g., economic, social, regulatory) and internal (e.g., culture, processes, strategy) factors is key to defining appropriate interface control measures. Note that, similarly to requirements, interface control may be iterative throughout the \index{Safety Assessment!Data}Data Safety Assessment. In particular, interface control may need to be amended to take account of implemented data safety mitigations\index{Mitigation}.

To form
a high-level understanding of each organization's risk,
the \gls{odr} assessment (\autoref{bkm:assessment}) can be used. The \gls{odr} assessment can be used at programme level to cover all \index{Stakeholder}Stakeholders, or used by each individual \index{Stakeholder}Stakeholder to determine the
risk
appropriate to their area.
Note however that care should be taken when adopting such a formulaic approach to risk assessment
to ensure that the resulting system can meet its cumulative requirements; that is, nothing has ``fallen through the cracks''. This is part of the requirements decomposition and verification / validation activities described later in the Data Safety Management Process.

Amongst other things, the \gls{odr} assessment includes: the severity of any potential accidents; organizational maturity; applicable legal and regulatory frameworks; and the size, complexity and novelty of the planned system. It results in a rating, from ODR0 (which corresponds to the lowest risk) to ODR4 (which corresponds to the highest risk). This rating provides an initial, top-level view of the magnitude of data-related risk. As such, it
could form
the basis for process tailoring; it also
gives an indication of the proportionate magnitude of effort that may be required in the management of data safety risks.

To allow tailoring to be applied in cases where the \gls{odr} assessment has not been explicitly conducted, the qualitative scale presented in \autoref{tab:Qualitative-ODR} is also used.

\begin{longtable}{|C{\dsiwgColumnWidth{0.21}}|L{\dsiwgColumnWidth{0.39}}|}
  \caption{Qualitative definition of \gls{odr}}
  \label{tab:Qualitative-ODR}
  \\\hline\TableHeadColourCX{\gls{odr} Rating} & \TableHeadColour{Qualitative Description}\\\hline
  \endfirsthead
  \caption[]{Qualitative definition of \gls{odr} (continued)}
  \\\hline\TableHeadColourCX{\gls{odr} Rating} & \TableHeadColour{Qualitative Description}\\\hline
  \endhead
  \multicolumn{2}{r}{\sl Continued on next page}
  \endfoot
  \endlastfoot
  {ODR4} & {High risk}\\\hline
  {ODR3} & \multirow{2}*{Medium risk}\\\cline{1-1}
  {ODR2} &\\\hline
  {ODR1} & {Low risk}\\\hline
  {ODR0} & {Very low risk}\\\hline
\end{longtable}

Note that, in the case of an ODR0, no further work is required.

In addition to the rating, the \gls{odr} can be used to facilitate the identification of key \index{Stakeholder}Stakeholders (i.e., those with an interest in the system) and necessary approvers (i.e., those who need to formally accept the system). Note that some approvers might be within the organization, while others may represent external bodies. Likewise, approvers could also be customers or regulatory authorities.

Part of the process of establishing the internal context involves understanding organizational culture. A short Data Safety Culture questionnaire has been developed (\autoref{bkm:culture}), which may help in this regard. This can be applied at an organization level or, more likely, within an individual project team. The questionnaire could also be used to highlight the importance of data safety related issues within a project team. In addition, before and after measurements could be taken to establish the effectiveness of data safety related training.

\paragraph{Describe the system context}
This activity is concerned with describing the system under analysis, as well as the key external influences on that system (examples of which include interfacing systems and human operators). There are obviously many aspects to this activity. For reasons of brevity only those aspects that are directly relevant to data safety are discussed here.

When describing the system it is often helpful to think in terms of producers and consumers of data. These may be external systems, or sub-systems, or a combination of both. In addition, it may be necessary to consider data supply chains, especially when there are a number of separate organizations involved. Note that the considerations discussed in this paragraph are intended to be addressed at a high level: the identification of specific pieces of data is a separate, but related, activity; the identification of required properties is another activity.

Also note that, as development progresses the system description is expected to be refined. This may enable the data safety system context to also be refined, supporting the \index{Safety Assessment!Data}Data Safety Assessment planning.

\paragraph{Plan the assessment}
This activity involves scheduling the phases associated with the Data Safety Management Process and acquiring the necessary resources to complete them. This also involves tailoring the generic process to meet the specific needs of a particular system development. 

Details of the planned assessment may be recorded in a \gls{dsmp}. This could also be used to capture the scope of the analyses and the associated context. Together, this information constitutes the first section of the \gls{dsmp} structure. Note that if a \gls{dsmp} is created, it would be expected to be updated with details from subsequent phases. Alternatively, if a Safety Management Plan has already been developed for the project, data safety aspects may augment the existing Safety Management Plan. 

Planning of the \index{Safety Assessment!Data}Data Safety Assessment requires some knowledge of the quantity and complexity of data that requires assessment. Therefore, the \gls{dsmp} (or Safety Management Plan) needs to be updated in subsequent phases once these details are known.

Planning of the assessment may be done through a procurement process. Procurers may wish to understand their potential supplier's understanding of data safety and plans to implement a \index{Safety Assessment!Data}Data Safety Assessment. A Data Safety Supplier Questionnaire (\autoref{bkm:maturity}) has been developed to support this analysis; this questionnaire may also be used for auditing purposes.

\paragraph{Identify \index{Artefact, Data}Data Artefacts}
\index{Artefact, Data}\Glspl{Data Artefact} are the key pieces of data that are generated, processed or consumed by the system, or used for training in its use or maintenance. They provide the foundation for the remaining phases of data-related risk management.

To support the identification of \index{Artefact, Data}\glspl{Data Artefact}, a wide variety of Data Categories have been enumerated. Cross-referencing these categories against the system description should highlight the relevant artefacts. 

Another way of identifying \index{Artefact, Data}\glspl{Data Artefact} involves considering the functions that the system performs and establishing the data that is required to support these.

A further option for confirming that all relevant \index{Artefact, Data}\glspl{Data Artefact} have been identified is to consider the different phases of the system lifecycle\index{Lifecycle!System}. This approach should help prevent an inappropriate focus on operational use of the system at the expense of, for example, artefacts associated with system test and evaluation.

\subsubsection{Tailoring}
The level of
\index{Stakeholder}stakeholder
interface control\index{Interface!organizational} required will depend heavily on the number of \index{Stakeholder}Stakeholders, complexity of their interactions, and the contractual controls already in place. Many programmes already require \glspl{icd}\index{Interface!Control Document} to be developed for systems or equipment. The level of detail required in other programme interface control plans may be used as a guide for the requirements of data safety interface control. 

The guidance in this document is general in nature and so it is anticipated that it will need to be tailored to align with the organization's attitude to risk, their existing processes and the relevant sector's regulatory environment. Thus the provided \gls{odr}, or a version customised to suit the organization's needs, may provide a structured approach to assessment.
The \gls{odr} assessment can be conducted at product line or individual product level, as appropriate for the organization. It is generally not recommended to conduct the \gls{odr} without the context of a system type.

It is expected that the \gls{odr} assessment will be of most utility for organizations that do not have significant safety engineering experience and that are operating in less well-regulated industrial sectors.
organizations with considerable experience in the development of safety-critical systems in heavily regulated environments
may also find it of use in defining the context from a data perspective and augmenting existing safety standards which have no explicit data considerations.
Note that if this assessment is not conducted, some high-level qualitative estimate of risk may still be required (e.g., to support process tailoring); likewise, there will also be a need to identify key \index{Stakeholder}Stakeholders and necessary approvers.

It is also expected that the Data Safety Culture questionnaire will be of most use for low-risk (i.e., ODR1) systems. In particular, it is expected that developers of higher risk systems will have extant processes to develop, maintain and monitor safety cultures,
although the data-oriented questionnaire could help inform those existing processes.

The approach of including data-related aspects within a Safety Management Plan is recommended for complex or highly safety-critical systems. In this case the structure of the Safety Management Plan may be maintained, with the \index{Safety Assessment!Data}Data Safety Assessment process being tailored to align to the overall \index{Safety Assessment}safety assessment process.

The questionnaire that helps establish the level of ``data maturity'' in potential suppliers is expected to be of most use when new organizational relationships are being formed. Conversely, it may offer little value in situations where both organizations are familiar with each other, they have worked on data-related projects together before and there are suitable audit / review arrangements in place.

\index{Artefact, Data}\Glspl{Data Artefact} may be defined at a number of levels. An artefact associated with a medical system could be described as ``patient data''. Alternatively, this could be split into smaller parts (e.g., ``blood group''). Generally speaking, the highest possible level consistent with the system description should be used; this prevents an excessively long list of artefacts being developed. If necessary, those artefacts where further detail is needed can be refined as part of an iterative process that is focused on key issues.

Not every Data Category\index{Data!Category|} will be relevant to every system. Furthermore, for low-risk (ODR1) systems it may be sufficient to simply consider the groupings of categories (e.g., ``context'', ``implementation'', etc.). Conversely, high-risk (ODR4) systems might need to consider every category\index{Category!Data}, even if this results in a conclusion that a specific category\index{Category!Data} is not relevant for the system in question.

A function-based approach to identifying \index{Artefact, Data}\glspl{Data Artefact} is likely to be enabled by design processes that also adopt a function-based perspective. If information from a function-based perspective is readily available then it should be used to support the identification of \index{Artefact, Data}\glspl{Data Artefact}. If this information is not readily available, it is recommended that it be generated for medium and high-risk systems (i.e., ODR2 to ODR4, inclusive).

Considering data across the system lifecycle\index{Lifecycle!System} is a relatively simple activity, which is applicable to all systems (i.e., ODR1 to ODR4, inclusive).

\subsection{Identify Risks}
\subsubsection{Overview}
This phase involves identifying sources of risk and understanding the potential consequences; it should result in a comprehensive list of risks. From a system development perspective, these activities are likely to be concurrent with the development of more detailed system designs.

\subsubsection{Activities}
There are three, complementary, activities that can be used to identify risks. There is also an activity associated with updating planning documents.

\paragraph{Review the general, historical perspective}
Some insight into potential risks may be gained by reviewing historical accidents and incidents, a collection of which is included in
\autoref{bkm:accidents} of this guidance document. It is expected that each domain would have its own catalogue of historical incidents, which can be consulted during a data-focused review.

\paragraph{Conduct a top-down approach}
If the system under consideration has clearly identified functions then data-related risks can be assessed by considering each function in turn and analysing what \index{Artefact, Data}\glspl{Data Artefact} and, more particularly, what Data Properties the function depends on.

If there are a limited number of safety-related functions, this is usually the simplest approach. This approach also has the advantage that it integrates well with other function-based, top-down approaches to assessing system safety. 

\paragraph{Conduct a bottom-up approach}
This approach starts from the \glspl{Data Artefact} and explores the effects of data errors. In this context an error is a situation where a required \index{Property!Data}Data Property is not exhibited. This may be achieved by a variety of methods, including a \gls{hazop}\index{HAZOP}.

\paragraph{Update planning documents}
Once the data safety risks have been identified, the Data Safety Management Plan (or Safety Management Plan)
requires review to determine if it needs
updating to take account of the quantity and complexity of the analysis and mitigation\index{Mitigation} activities needed to address the risks. While the \gls{dsmp} may be updated throughout the \index{Safety Assessment!Data}Data Safety Assessment, updates may not be required for all projects.

\subsubsection{Tailoring}
The general, historical perspective review is a simple activity that does not require significant resources. Hence, it is recommended for all systems, regardless of risk level.

When conducting a bottom-up approach, it may not be appropriate to explicitly consider every possible property for every single artefact. In particular, for low-risk (ODR1) and medium-risk (ODR2 / ODR3) systems some form of tailoring may be expected: this may, for example, take the form of pre-selecting the properties that are most relevant, or limiting the layer of abstraction at which the system is considered.

Tailoring of the bottom-up approach may also be appropriate for some high-risk (ODR4) systems, but in this case an explicit argument that the tailoring has not adversely affected system safety would be expected. Furthermore, the risk identification process for high-risk (ODR4) systems is expected to be a highly structured affair. To support this, a number of data-related \gls{hazop}\index{HAZOP} Guidewords have been determined
and are presented in \autoref{bkm:guidewords}.

The top-down and bottom-up approaches provide different perspectives when attempting to identify data-related risks. For low-risk (ODR1) systems it may be appropriate to consider just one of these perspectives. Conversely, both perspectives would be expected to be considered (to some degree) for medium-risk (ODR2 / ODR3) and high-risk (ODR4) systems.
%
%\clearpage %Manual page break
\subsection{Analyse Risks}
\subsubsection{Overview}
This part of the risk management process involves developing an understanding of the consequences and likelihood of each risk. From the perspective of safety-critical and safety-related systems this understanding allows System (or Safety) Integrity Levels or \index{Assurance Level!Development}Development Assurance Levels to be determined. Likewise, this understanding should be used to allocate
\glspl{dsal}\index{Assurance Level!Data}\index{DSAL|see{Assurance Level, Data}}.
%
%\vspace{-8pt}%Cheat needed to avoid table bleeding over to next page. Only needed since file indexed!
%
\subsubsection{Activities}
There are two activities associated with this phase.
%
\paragraph{Establish \glspl{dsal}}\index{Assurance Level!Data|textbf}
\label{bkm:DSAL-table-section}
The key activity in this phase is to establish the (untreated) likelihood and severity of each risk identified in the preceding phase.

To analyse risks and, more particularly, to align data safety with other risk management processes, there is a need to overcome problems stemming from the use of the term ``likelihood'' in situations where there may be no failure rates. For this reason the \gls{dsal}\index{Assurance Level!Data} was developed. The \gls{dsal} metric is not a statistical measure of likelihood, or a literal numeric measure of \index{Integrity Property}integrity. Instead, the \gls{dsal} metric is an indicator for the level of rigour that an assurance argument requires. As such, \glspl{dsal} share a common theoretical basis with concepts like Item \index{Assurance Level!Development}Development Assurance Levels \cite{citation:arp4754a2010guidelines} and development process systematic capability \cite{citation:iec615083}.

\glspl{dsal} are measured on a scale of DSAL0 (lowest-assurance) to DSAL4 (highest-assurance). They are typically allocated as indicated in \autoref{tab:DSAL-risk-matrix}.

\begin{longtable}{|C{\dsiwgColumnWidth{0.2}}|C{\dsiwgColumnWidth{0.2}}|C{\dsiwgColumnWidth{0.2}}|C{\dsiwgColumnWidth{0.2}}|}
  \caption{\gls{dsal} "risk" matrix}
  \label{tab:DSAL-risk-matrix}
  \\\hline
  \TableHeadColour{} & \multicolumn{3}{c|}{\TableHeadColourCX{Likelihood}}\\\cline{2-4}
  \multirow{-2}*{\TableHeadColourCX{Severity}} & \TableDimColourCX{ High} & \TableDimColourCX{Medium} & \TableDimColourCX{Low}\\\hline
  \endfirsthead
  \caption[]{\gls{dsal} "risk" matrix (continued)}
  \\\hline
  \TableHeadColour{} & \multicolumn{3}{c|}{\TableHeadColourCX{Likelihood}}\\\cline{2-4}
  \multirow{-2}*{\TableHeadColourCX{Severity}} & \TableDimColourCX{High} & \TableDimColourCX{Medium} & \TableDimColourCX{Low}\\\hline
  \endhead
  \multicolumn{4}{r}{\sl Continued on next page}
  \endfoot
  \endlastfoot
  Minor & DSAL1 & DSAL0 & DSAL0\index{Assurance Level!Data}\\\hline
  Moderate & DSAL2 & DSAL1 & DSAL0\index{Assurance Level!Data}\\\hline
  Significant & DSAL3 & DSAL2 & DSAL1\index{Assurance Level!Data}\\\hline
  Major & DSAL4 & DSAL3 & DSAL2\index{Assurance Level!Data}\\\hline
  Catastrophic & DSAL4 & DSAL4 & DSAL3\index{Assurance Level!Data}\\\hline
\end{longtable}

Definitions for Severity and Likelihood associated with \autoref{tab:DSAL-risk-matrix} may be customised for a specific application. A default approach to the assessment of Likelihood is presented in
\hyperref[bkm:Establishing-DSALS]{section}\index{Assurance Level!Data}~\ref{bkm:Establishing-DSALS} %Clunky, but \autoref inserted "subsubsection"
and \autoref{tab:Likelihood}, while default definitions for Severity are presented in \autoref{tab:Severity}.

Although this allocation of \glspl{dsal}\index{Assurance Level!Data} is typically used, it is acknowledged that there are some situations where a different allocation matrix may be more appropriate. Hence, it may be appropriate for a tailored allocation matrix to be used. Regardless of whether tailoring is used, the matrix should be reviewed and confirmed as being suitable for the intended application.

As their name suggests, \glspl{dsal} are focused on safety concerns. However, the framework of \index{Artefact, Data}\glspl{Data Artefact}, \index{Property!Data}Data Properties, and so on, developed in this document could also be applied to other concerns. It could, for example, be used for to control data-related financial risks, or data-related reputational risks. In these types of approach, the severity terms would obviously relate to financial and reputational consequences, rather than safety ones.

It is possible that the additional understanding developed during this part of the process may mean some previously identified \glspl{Data Artefact} are no longer of consequence; similarly, it is possible that this process may identify additional artefacts or a need to refine the description of existing artefacts.

\paragraph{Analyse \glspl{dsal} as part of system safety activities}\index{Assurance Level!Data}
\label{bkm:activities:analyse:partofsystemsafetyactivities}
Allocating a \gls{dsal}\index{Assurance Level!Data} is a significant part of controlling data safety risks, but it is not the only part. It is important that \glspl{dsal} are considered as part of wider system safety activities, rather than being viewed as a separate item.

For medium-risk (ODR2 / ODR3) and high-risk (ODR4) systems it is likely that \index{Integrity Property}integrity, or assurance, levels will be calculated from perspectives other than data safety. Possible examples include Item / Function \index{Assurance Level!Development}Development Assurance Levels from Aerospace Recommended Practice (ARP) 4754A \cite{citation:arp4754a2010guidelines} and Safety Integrity Levels from IEC 61508 \cite{citation:iec615083}.
Where such an approach is used, the mapping to \glspl{dsal}\index{Assurance Level!Data} should be included within the \gls{dsmp}.

This activity involves comparing \glspl{dsal}\index{Assurance Level!Data} with these other integrity, or assurance, levels. Assuming a typical scenario of a system processing or manipulating data flowing through it, there are two cases to consider:

\begin{enumerate}
  \item Can the data affect the software? In particular, this question is concerned with whether the data can affect the software such that the safe operation of the system is jeopardised. Obviously an ideal system would be able to handle any data fed into it safely without problems, but this is often not the case. An example might be a legacy system which has limited error checking and so may fail in unsafe ways if fed with data which is outside of the expected range. Formally this question can be stated as: \dsiwgTextIT{Given a system containing software written to a particular \index{Assurance Level!Software}Software Assurance Level (which may be none), what should the \gls{dsal}\index{Assurance Level!Data} of the processed data be to preserve correct operation of the system?}
  \item Can the software affect the data? In particular, this question is concerned with whether the system's software can affect the data being processed or manipulated in such a way that \index{Property!Data}Data Properties that are important for safety might be lost. Some examples might be systems which transform messages, losing any associated checksum protections, thereby possibly affecting the \index{Integrity Property}integrity of the data within the message; as a minimum, this removes a means of checking data \index{Integrity Property}integrity. Another example might be a system that can delay data flowing through it, (e.g., due to buffering) when timely delivery of the data is critical. Formally, this question can be stated as: \dsiwgTextIT{Given data at a particular \gls{dsal}\index{Assurance Level!Data} what should the \index{Assurance Level!Software}Software Assurance Level of the software in the system be in order to preserve the \gls{dsal}\index{Assurance Level!Data} of the data?}
\end{enumerate}

\subsubsection{Tailoring}\label{bkm:activities:analyse:tailoring}
\glspl{dsal}\index{Assurance Level!Data} can be applied at different levels and to different constructs. For example, in the case of simple, low-risk systems it may be appropriate to apply a single \gls{dsal} to an entire system. Alternatively, it may be appropriate to apply \glspl{dsal} to sub-systems, for example, to match the level at which other integrity, or assurance, levels have been determined. 

Another option is to apply \glspl{dsal}\index{Assurance Level!Data} to \glspl{Data Artefact} rather than directly to risks. This approach has the advantage that \index{Treatment!Risk}treatments are often related to artefacts; it can work well where there is a simple relationship between artefacts and risks.

%\clearpage %Manual page break
\subsection{Evaluate and Treat Risks}
\subsubsection{Overview}
This phase involves deciding, at a generic level, what action (if any) should be taken for each of the risks identified in preceding phases. This decision will be influenced by the organization's risk appetite and other factors determined as part of the Establish Context phase. From some perspectives it may seem strange that requirements are identified at such a late phase. This is a consequence of explicitly linking \index{Safety Requirement!Data}Data Safety Requirements to risks associated with \index{Property!Data}Data Properties of \index{Artefact, Data}\glspl{Data Artefact}, and the use of \index{Assurance Level!Data}Data Safety Assurance Levels to describe levels of rigour. 

This phase also involves identifying, implementing and verifying \index{Treatment!Risk}treatments for the risks emerging from the previous phase. Part of verifying the \index{Treatment!Risk}treatment involves checking technical details of the chosen approach; another part involves re-assessing the post-treatment risk to determine whether it is now acceptable.

\subsubsection{Activities}
This phase involves reviewing each risk, including the associated \gls{dsal}\index{Assurance Level!Data} and determining the appropriate \index{Response}response. Essentially, this phase aims to answer the question: can we accept this risk or does some action need to be taken? This is likely to require discussion amongst a number of \index{Stakeholder}Stakeholders. From a system safety perspective, there is nothing intrinsically special about data-related risks. Hence, it is recommended that evaluation of data-related risks be conducted alongside the evaluation of other system risks, as part of an organization's standard risk evaluation process.

Risks may be managed in different ways, for example:

\begin{itemize}
  \item \nonumparagraph{Avoid}
	
	Risk avoidance can be employed where a risk can be eliminated by using different approaches to the design and / or operation of the system. It may also be the case that very significant risks cannot be adequately treated and the only option is to avoid the untenable risk by not proceeding with the project.

  \item \nonumparagraph{Accept} 
	
    For low likelihood and low severity risks (e.g., those ranked as DSAL0)\index{Assurance Level!Data}, where the cost of further risk reduction is judged to be unacceptable, the risk may be accepted as-is and managed as such. Appropriate justification\index{Justification, Safety} is likely to be required for acceptance of risks ranked higher than DSAL0\index{Assurance Level!Data}.
\clearpage   
  \item \nonumparagraph{Transfer} 
	
	In this case ownership of the risk's consequence is transferred to another organization. This can be achieved, for example, by taking out an insurance policy. It is important that any such risk transfers are formally documented, understood and
	agreed
	by both parties.

  \item \nonumparagraph{Treat} 
	
	In this case there is a desire to reduce the risk. This can be achieved by reducing the severity, or the likelihood or both. Choosing this \index{Response}response often involves having an outline view of how the risk may be reduced. 
\end{itemize}

In addition to deciding on and documenting the appropriate \index{Response}response to each risk, this phase also includes gaining approval for these decisions.

In cases when a decision is made to treat a risk, suitable methods and approaches should be identified. To assist in this process, a range of potential methods and approaches are included in this guidance. These are mapped against \glspl{dsal}\index{Assurance Level!Data}, \index{Property!Data}Data Properties and a selection of lifecycle\index{Lifecycle!Data} data categories.

Once a \index{Treatment!Risk}treatment strategy has been established and implemented there is, of course, a need to determine whether the expected risk reduction has been achieved. Equivalently, there is a need to consider whether the residual risk may now be accepted (or whether another one of the \index{Response}responses identified above is necessary).

\subsubsection{Tailoring}
It is expected that records will be kept of the discussions that occur as part of risk evaluation. For low-risk (ODR1) systems, this may be in the form of a brief memo. Conversely, for high-risk (ODR4) systems, a detailed, structured record, which is placed under formal control, may be expected; in this case these discussions may be recorded as part of the system's Hazard Log or as part of a Data Safety Management Plan.

As outlined in
\hyperref[bkm:activities:analyse:tailoring]{section}~\ref{bkm:activities:analyse:tailoring}, %Clunky, but \autoref inserted "subsubsection"
\glspl{dsal}\index{Assurance Level!Data} can be applied at varying levels of abstraction. For small-scale or low-risk (ODR1) systems it may be appropriate to consider \index{Treatment!Risk}treatments at higher levels of system abstraction. For example, this could involve applying a single \gls{dsal} to a sub-system or even to the system in its entirety and implementing risk \index{Treatment!Risk}treatment techniques at that level. The latter approach could be appropriate where the data in the system interacts in complex ways and the associated safety risk does not warrant a detailed investigation of these interactions.

A significant amount of tailoring is implicit in the way the tables of methods and approaches are constructed. At best a method / approach may be Highly Recommended as a way of maintaining a required \index{Property!Data}Data Property at a given \gls{dsal}\index{Assurance Level!Data}. In addition, the tables are not exhaustive; additional, or alternative, methods and approaches can be used.
