%================================================================================
%       Safety Critical Systems Club - Data Safety Initiative Working Group
%================================================================================
%                       DDDD    SSSS  IIIII  W   W   GGGG
%                       D   D  S        I    W   W  G   
%                       D   D   SSS     I    W W W  G  GG
%                       D   D      S    I    WW WW  G   G
%                       DDDD   SSSS   IIIII  W   W   GGG
%================================================================================
%               Data Safety Guidance Document - LaTeX Source File
%================================================================================
%
% Description:
%   This file contains all of the LaTeX macros to control the template for the
%   Data Safety Guidance Document.
%
% !!WARNING!!
%   CHANGING THE CONTENT OF THIS PARTICULAR FILE CAN SERIOUSLY AFFECT THE
%   APPEARANCE AND LAYOUT OF THE GENERATED DOCUMENT, SO PLEASE BE CAREFUL.
%
% Label formats:
%   bkm: ==>      Reference to another part of this document
%   citation: ==> Reference to an external document listed in bibTex
%   fig: ==>      Reference to a figure within this document
%   ftn: ==>      Reference to a footnote within this document
%   tab: ==>      Reference to a table within this document
%
% Notes:
%   All commands, styles, colours, etc. defined within this template shall begin
%   with the prefix of 'dsiwg' to distinguish them from built-in TeX keywords or
%   those in third-party packages.
%
%================================================================================
% Configuration for different versions of the document
%================================================================================

% In the main.tex, before including this template, include the following two lines.
% Comment out the options that are not wanted

%\def\withCovers{}
%\def\withChangebars{} %Markers are \cbstart and \cbend. Remember to delete markers for previous changes before starting update.

%================================================================================
%Version 2.0 was US Letter paper with a default font of 10pt, this may change
%================================================================================
\documentclass[letterpaper,10pt,twoside]{article}

% May be able to remove in future, when pdflatex default output becomes pdf 1.7 or greater
\pdfminorversion=7 % Required to read darkknowns.pdf at v1.7, but default from pdflatex is 1.5

%================================================================================
%Set the default font family and series for the document to 'Open Sans Light'
%================================================================================
\usepackage[utf8]{inputenc}
\usepackage[T1]{fontenc}
\usepackage[default,defaultsans]{opensans}% On TeXLive 2018, "default" was enough. Need "defaultsans" from TeXLive 2019
\usepackage{textcomp}           %Required for complete font support
\renewcommand{\seriesdefault}{l}% Select Open Sans Light as the default font

%================================================================================
%Load all other required packages
%================================================================================
\usepackage[
  title,%
  titletoc%
]{appendix}                     %Used to add appendices into the ToC
\usepackage{tocvsec2}           %Allow suppression of selected entries within TOC
%--------------------------------------------------------------------------------
\usepackage{calc}               %Used for infix calculations, e.g. column widths 
\usepackage{color}              %Used to set foreground and background colours
\usepackage{colortbl}           %Used to add colour to LaTeX tables
\usepackage{enumitem}           %Used to control the different list environments
\usepackage{fancyhdr}           %Used to define the header and footer layout
\usepackage{float}              %Used to force specific placement of images
\usepackage[hang]{footmisc}     %Used to make footnotes a numbered list
%--------------------------------------------------------------------------------
\ifx\withChangebars\undefined
% Shortcut to create version without changebars
\newcommand{\cbstart}{}
\newcommand{\cbend}{}
\else
\usepackage{changebar}          %Markers are \cbstart and \cbend.
\fi
%--------------------------------------------------------------------------------
\usepackage[
  dvips =false,%
  pdftex=false,%
  vtex  =false,%
  bottom=1.5cm,%
  left  =2.5cm,%
  right =2.5cm,%
  top   =1.5cm,%
  head  =18pt  %
]{geometry}                     %Used to set-up the page layout and margins
%--------------------------------------------------------------------------------
\usepackage[pdftex]{graphicx}   %Used to insert diagrams into the document
%--------------------------------------------------------------------------------
%\usepackage[
%  acronyms,%
%  toc
%]{glossaries}                   %Used to create and print the glossary - very sensitive to package order
%--------------------------------------------------------------------------------
\usepackage[none]{hyphenat}     %Used to prevent hyphenation in the document
\usepackage{longtable}          %Used for tables that break across pages
\usepackage{caption}            %Adds longtable* which is needed to create longtables that do not increment the table counter
\usepackage{multirow}           %Used to allow multiple row spanning in tables
\usepackage{pifont}             %Used to print ZipfDingbats characters (\ding)
\usepackage[many]{tcolorbox}    %Used to place coloured boxes around text

\ifx\withToDo\undefined
% Shortcut to create version without todo notes
\newcommand{\todo}[1]{}
\newcommand{\listoftodos}{}
\else
\usepackage[textwidth=2cm]{todonotes} %Allows multi-line markup of text which needs to be amended/reviewed ie, yellow highlighter
\fi

\usepackage{pdfpages}           %Handy way to insert covers
\usepackage{wrapfig}            %Wrap text around a figure
\usepackage{imakeidx}

\usepackage[framemethod=tikz]{mdframed}           %Put text in a box that can break across pages
\newmdenv[frametitle=Response of the AI,%
  roundcorner=10pt,
linewidth=2pt]{aibox} % Environment to display AI-generated text
%--------------------------------------------------------------------------------
% Set up indexes.
% Page style gets upset for indexes, unless we set it explicitly
%
\makeindex[name=locationidx,title=Index of Locations,intoc]
\makeindex[title=Index,intoc]
\indexsetup{
  othercode={%
    \thispagestyle{Standard}%
  }
}
%--------------------------------------------------------------------------------
%The hyperref is best loaded last to avoid conflicts, e.g. with glossaries
%TGR: the glossaries package documentation says it must be loaded after the hyperref package.
%--------------------------------------------------------------------------------
\ifx\withBlueHyperlinks\undefined
\usepackage[                    %Could use \hypersetup but easier to do it here
  bookmarksnumbered,%
  pdftex,%
  pdfpagelayout=OneColumn,% "OneColumn" produces a single continous scrolling viewer display; was previously set to "TwoColumnRight" for two pages side by side with odd pages on the right.
  pdfpagelabels,%
  hidelinks,% Make the links visually just normal text
  linktoc=page
]{hyperref}                     %Creates property fields & hyperlinks in the PDF - For details see http://www.tug.org/applications/hyperref/manual.html#x1-110003.7
\else
\usepackage[                    %Could use \hypersetup but easier to do it here
  bookmarksnumbered,%
  pdftex,%
  pdfpagelayout=OneColumn,% "OneColumn" produces a single continous scrolling viewer display; was previously set to "TwoColumnRight" for two pages side by side with odd pages on the right.
  pdfpagelabels,%
  colorlinks=true,% Select coloured text, rather than coloured boxes around links
  linkcolor=blue,          % color of internal links (change box color with linkbordercolor)
  citecolor=blue,        % color of links to bibliography
  filecolor=blue,      % color of file links
  urlcolor=blue,           % color of external links
  linktoc=page
]{hyperref}                     %Creates property fields & hyperlinks in the PDF - For details see http://www.tug.org/applications/hyperref/manual.html#x1-110003.7
\fi

%================================================================================
%Create some new commands to quickly change local font style
%================================================================================
\DeclareRobustCommand\ebseries{\fontseries{eb}\selectfont}
\DeclareRobustCommand\sbseries{\fontseries{sb}\selectfont}
\DeclareRobustCommand\ltseries{\fontseries{l}\selectfont}
\DeclareRobustCommand\clseries{\fontseries{cl}\selectfont}
\DeclareRobustCommand\regseries{\fontseries{m}\selectfont}

\DeclareTextFontCommand{\dsiwgTextEB}{\ebseries}    %Extra-Bold typeface
\DeclareTextFontCommand{\dsiwgTextSB}{\sbseries}    %Semi-Bold typeface
\DeclareTextFontCommand{\dsiwgTextLT}{\ltseries}    %Light typeface
\DeclareTextFontCommand{\dsiwgTextCL}{\clseries}    %Condensed-light typeface
\DeclareTextFontCommand{\dsiwgTextREG}{\regseries}  %Regular typeface

\DeclareTextFontCommand{\dsiwgTextBF}{\sbseries}    %What 'bold-face' looks like
\DeclareTextFontCommand{\dsiwgTextIT}{\itshape}     %What 'italic' looks like

%
%Create a new environment to change the font shape for large blocks of text.
%Using begin/end is more obvious than just using one of the above commands.
%
\newenvironment{dsiwgBold}{\sbseries}{}
\newenvironment{dsiwgItalic}{\itshape}{}

%
%Unfortunately the Open Sans font doesn't support a typewriter style in 'light'
%series, so we have to declare a special version to avoid warnings when compiling
%the document.
%
% For manual use:
\newcommand{\dsiwgTextTT}[1]{\usefont{T1}{cmtt}{m}{n}{#1}}
%
% To make \url use the right font
% We need these next two lines, but they fail to load T1/cmtt/m/n, so we put up with warning for now
%\DeclareFontFamily{T1}{cmtt}{\hyphenchar\font=-1}
%\DeclareFontShape{T1}{cmtt}{l}{n}{ <-> ssub * cmtt/m/n }{}

%
%Set the character to be used for the check box on the ODR forms, etc.
%
\newcommand{\dsiwgCheckBox}{\large{\ding{111}}}

%================================================================================
%Space things out a bit
%================================================================================
\linespread{1.2}

%================================================================================
%Mark intentionally blank pages as such
%================================================================================
\makeatletter
\def\dsiwg@intblankpage{%
	\clearpage% this line has no effect if invoked from dsiwg@cleardoublepage
	\null\vfil
	\centerline{This page is intentionally blank}%
	\newpage
        \phantomsection% Required to move page counter in \label associated with following \section point to correct page
}

\def\dsiwg@cleardoublepage{%
    \clearpage%needed here so that we test AFTER the page throw
	\ifodd\c@page\else
		\dsiwg@intblankpage
	\fi
}
\let\cleardoublepage=\dsiwg@cleardoublepage
\makeatother

%================================================================================
%Define the colours to be used for this template style
%================================================================================
%
%Document accent colour for headings, etc. (set to a shade of blue)
%
%\definecolor{dsiwgAccentColour}{RGB}{233,33,33} %Red - v2.0
\definecolor{dsiwgAccentColour}{RGB}{0,102,255} %Blue - v3.0
\definecolor{dsiwgDimColour}{RGB}{51,153,255} %Pale Blue - v3.6
%
%Colours to be used for the objectives list box at the start of each section 
%
%\definecolor{dsiwgObjectiveBackgroundColour}{RGB}{233,15,12}
%\definecolor{dsiwgObjectiveFrameColour}{RGB}{233,132,132}

%================================================================================
%Assign colours to the various document elements
%================================================================================
%
%Set the colours for all of the section headings
%
\colorlet{dsiwgSectionColour}{dsiwgAccentColour}
\colorlet{dsiwgSubSectionColour}{dsiwgSectionColour}
\colorlet{dsiwgSubSubSectionColour}{dsiwgSectionColour}
\colorlet{dsiwgParagraphColour}{dsiwgSectionColour}

%
%Set the colours for all four levels of enumerated list
%
\preto\labelenumi{\color{dsiwgAccentColour}}
\preto\labelenumii{\color{dsiwgAccentColour}}
\preto\labelenumiii{\color{dsiwgAccentColour}}
\preto\labelenumiv{\color{dsiwgAccentColour}}

%
%Set the colours for all four levels of bulleted list
%Also make the top-level bullet large
%
%\preto\labelitemi{\large\color{dsiwgAccentColour}}
\renewcommand\labelitemi{\large\color{dsiwgAccentColour}$\bullet$}
\preto\labelitemii{\color{dsiwgAccentColour}}
\preto\labelitemiii{\color{dsiwgAccentColour}}
\preto\labelitemiv{\color{dsiwgAccentColour}}

%
%Set the background and edge colours for the quotes
%
\definecolor{dsiwgQuoteBackColour}{gray}{0.95}
\definecolor{dsiwgQuoteFrameColour}{gray}{0.65}

%
%Set up colour for TBD highlighting - stuff to be fixed prior to release
%
%\definecolor{dsiwgTbdColour}{rgb}{1,1,0}
%\newcommand{\dsiwgTbd}[1]{{\todo[color=dsiwgTbdColour,inline]{#1}}}
\setlength{\marginparwidth}{2cm}%Keeps todo note in margin

%================================================================================
% Enhancements to allow fixed width columns, with left, centred or right aligned
% text (the usual m option only allows justified, creating underfull hboxes)
%================================================================================
\newcolumntype{L}[1]{>{\raggedright\let\newline\\\arraybackslash\hspace{0pt}}m{#1}}
\newcolumntype{C}[1]{>{\centering\let\newline\\\arraybackslash\hspace{0pt}}m{#1}}
\newcolumntype{R}[1]{>{\raggedleft\let\newline\\\arraybackslash\hspace{0pt}}m{#1}}

%================================================================================
%Format the quotes at the beginning of each section. This takes two arguments:
%1) the quote, 2) the author (can be empty if the quote is not attributed)
%================================================================================
\newcommand\dsiwgSectionQuote[2]{
  \begin{center}
    \begin{tcolorbox}
      [ boxsep=0mm,
        colback=dsiwgQuoteBackColour,
        colframe=dsiwgQuoteFrameColour,
        bottomrule=0mm,
        toprule=0mm,
        leftrule=4pt,
        rightrule=4pt,
        sharp corners,
        width=0.89\linewidth
      ]
      \begin{center}
        \dsiwgTextIT{#1}
        \ifx&#2&
          %Empty field, do nothing.
        \else
          \\\dsiwgTextBF{\dsiwgTextIT{{#2}}}
        \fi
      \end{center}
    \end{tcolorbox}
  \end{center}
}

\newcommand\dsiwgColumnWidth[1]{#1\linewidth-2\tabcolsep-1.25\arrayrulewidth}

%================================================================================
%Create a new environment for the objectives box used at the start of a section.
%================================================================================
\usetikzlibrary{shadings}
\newenvironment{dsiwgObjectiveList}{\begin{enumerate}[label=\arabic*.\color{white}] \color{white} \sbseries \slshape}{\end{enumerate}}
\tcolorboxenvironment{dsiwgObjectiveList}
{%
  enhanced jigsaw,
  boxsep=0mm,
  colframe=dsiwgObjectiveFrameColour,
  boxrule=4pt,
  sharpish corners,
  drop shadow=black,
  interior style={left color=dsiwgObjectiveBackgroundColour!95!white,right color=dsiwgObjectiveBackgroundColour!95!white,middle color=dsiwgObjectiveBackgroundColour!75!white,opacity=0.75},
  width=0.89\textwidth
}

%================================================================================
%Set-up the table header colour scheme and fonts
%================================================================================

%
%There are so many ``centered'' or ragged entries that it was worth adding
%special commands for these.
%
\newcommand\TableHeadColour[1]{\cellcolor{dsiwgAccentColour}{\ebseries\small\color{white}{#1}}}
\newcommand\TableHeadColourC[1]{\centering\TableHeadColour{#1}}% Version with contents centred
\newcommand\TableHeadColourR[1]{\raggedleft\TableHeadColour{#1}}% Version with contents raggedleft
\newcommand\TableDimColour[1]{\cellcolor{dsiwgDimColour}{\ebseries\small\color{white}{#1}}}
%
%Special versions with contents centred, and \\ bug correction for right hand column
%Modification of above form doesn't work - the position of the \arraybslash seems to be critical
%
\newcommand\TableHeadColourCX[1]{\cellcolor{dsiwgAccentColour}\centering\arraybslash{\ebseries\small\color{white}#1}}
\newcommand\TableDimColourCX[1]{\cellcolor{dsiwgDimColour}\centering\arraybslash{\ebseries\small\color{white}#1}}

%
%Set table fonts here to avoid setting on every table cell
%
\makeatletter
% Thanks to Axel Sommerfeldt [2007/01/07] for this macro:
\providecommand*\AtBeginEnvironment[1]{%
  \@ifundefined{#1}%
               {\@latex@error{Environment #1 undefined}\@ehc
                 \@gobble}%
               {\@ifundefined{ABE@env@#1}%
                 {\expandafter\let\csname ABE@env@#1\expandafter\endcsname
                   \csname #1\endcsname
                   \expandafter\let\csname ABE@hook@#1\endcsname\@empty
                   \@namedef{#1}{\@nameuse{ABE@hook@#1}\@nameuse{ABE@env@#1}}}%
                 {}%
                 \expandafter\g@addto@macro\csname ABE@hook@#1\endcsname}}
\@onlypreamble\AtBeginEnvironment
\makeatother

%
%Whenever we create a longtable set the font size
%
\AtBeginEnvironment{longtable}{\small}

%================================================================================
%Set-up some text styles
%================================================================================

%
%Colour for numbers in section, subsection and subsubsection
%
\newcommand\textstyleNumberingSymbols[1]{\textsf{\textcolor{dsiwgAccentColour}{#1}}}
\newcommand\textstyleBulletSymbols[1]{{\normalsize\selectfont\textsf{\textbf{\textcolor{dsiwgAccentColour}{#1}}}}}
\newcommand\textstyleCitation[1]{\textit{#1}}
\newcommand\textstyleFootnoteanchor[1]{{\normalsize\selectfont\textsf{\textmd{\textsuperscript{#1}}}}}

%================================================================================
%Set up spacing heading/section spacing, as per original Version 2.0 template
%================================================================================
\makeatletter
  %Heading Level 1
  \renewcommand\section%
  {%
    \@startsection{section}{1}{0pt}{4pt}{0.1mm}%
    {\cleardoublepage\pagestyle{Standard}\thispagestyle{FirstPage}\raggedright\normalfont\LARGE\regseries\normalcolor\color{dsiwgSectionColour}}%
  }

  %Heading Level 2
  \renewcommand\subsection%
  {%
    \@startsection{subsection}{2}{0pt}{18pt}{0.1mm}%
    {\normalfont\Large\regseries\normalcolor\color{dsiwgSubSectionColour}}%
  }

  %Heading Level 3
  \renewcommand\subsubsection%
  {%
    \@startsection{subsubsection}{3}{0pt}{18pt}{0.1mm}%
    {\normalfont\large\regseries\normalcolor\color{dsiwgSubSubSectionColour}}%
  }

  %Heading Level 4
  \renewcommand\paragraph%
  {%
    \@startsection{paragraph}{4}{0pt}{18pt}{0.1mm}%
    {\normalfont\normalsize\regseries\normalcolor\color{dsiwgParagraphColour}}%
  }
	
\makeatother

%A Non-numbered version of paragraph
\newcommand\nonumparagraph[1]{{\normalfont\normalsize\regseries\normalcolor\textcolor{dsiwgParagraphColour}{#1}}}


%================================================================================
%Set-up the headings and outline numbering
%================================================================================
\makeatletter
  \renewcommand\@seccntformat[1]{\csname @textstyle#1\endcsname{\csname the#1\endcsname}\csname @distance#1\endcsname}
  \setcounter{secnumdepth}{4}
  \newcommand\@distancesection{\hspace{1cm}}
  \newcommand\@textstylesection[1]{\protect\makebox[0cm][l]{\textstyleNumberingSymbols{#1}}}
  \newcommand\@distancesubsection{\hspace{1.27cm}}
  \newcommand\@textstylesubsection[1]{\protect\makebox[0cm][l]{\textstyleNumberingSymbols{#1}}}
  \newcommand\@distancesubsubsection{\hspace{1.27cm}}
  \newcommand\@textstylesubsubsection[1]{\protect\makebox[0cm][l]{\textstyleNumberingSymbols{#1}}}
  \newcommand\@distanceparagraph{\hspace{1.52cm}}\newcommand\@textstyleparagraph[1]{\protect\makebox[0cm][l]{\textstyleNumberingSymbols{#1}}}
\makeatother

\makeatletter
  \newcommand\arraybslash{\let\\\@arraycr}
\makeatother
\raggedbottom

%================================================================================
%Set-up the paragraph styles
%================================================================================
\renewcommand\familydefault{\sfdefault}

\newenvironment{styleTextbody}{\setlength\leftskip{0cm}\setlength\rightskip{0cm}\setlength\parindent{0cm}\setlength\parfillskip{0pt plus 1fil}\setlength\parskip{0.349cm plus 0.0349cm}\leavevmode\normalfont\normalsize\normalcolor\ignorespaces}{\unskip\vspace{0cm plus 1pt}\par}

\newenvironment{styleMainTitle}{\clearpage\setcounter{page}{1}\pagestyle{FirstPageFrontMatter}
  \thispagestyle{FirstPageTitle}
  \setlength\leftskip{3.551cm}\setlength\rightskip{2cm}\setlength\parindent{0cm}\setlength\parfillskip{0pt plus 1fil}\setlength\parskip{0cm plus 1pt}\leavevmode\normalfont\huge\normalcolor\color{dsiwgAccentColour}\ignorespaces}{\unskip\vspace{0.28cm plus 0.027999999cm}\par}

\newenvironment{styleTitlePageAuthor}{\setlength\leftskip{3.551cm}\setlength\rightskip{5.001cm}\setlength\parindent{0cm}\setlength\parfillskip{0pt plus 1fil}\setlength\parskip{0cm plus 1pt}\leavevmode\normalfont\Large\normalcolor\color{dsiwgAccentColour}\ignorespaces}{\unskip\vspace{0cm plus 1pt}\par}

\newenvironment{styleTitlePageDate}{\setlength\leftskip{3.551cm}\setlength\rightskip{2cm}\setlength\parindent{0cm}\setlength\parfillskip{0pt plus 1fil}\setlength\parskip{0.28cm plus 0.027999999cm}\leavevmode\normalfont\normalsize\normalcolor\sffamily\mdseries\color{dsiwgAccentColour}\ignorespaces}{\unskip\vspace{0cm plus 1pt}\par}

\newenvironment{styleQuotation}{\setlength\leftskip{0cm plus 1fil}\setlength\rightskip{0cm plus 1fil}\setlength\parindent{0cm}\setlength\parfillskip{0pt}\setlength\parskip{0.349cm plus 0.0349cm}\leavevmode\normalfont\small\normalcolor\ignorespaces}{\unskip\vspace{0cm plus 1pt}\par}

%
%Footnote rule
%
\setlength{\skip\footins}{0.119cm}
\renewcommand\footnoterule{\vspace*{-0.018cm}\setlength\leftskip{0pt}\setlength\rightskip{0pt plus 1fil}\noindent\textcolor{black}{\rule{0.25\columnwidth}{0.018cm}}\vspace*{0.101cm}}

%================================================================================
%Define the coloured blob to go in the footer, near the page number
%================================================================================
%
%The blob itself
%
\newcommand{\blob}{\textcolor{dsiwgAccentColour}{\rule[-1em]{0.5em}{4em}}}

%
%Make it take up zero vertical space and set up header and footer offsets for it
%
\newcommand{\tlap}[1]{\mbox{\vbox to 0pt{\vss\hbox{#1}}}}
\newlength\OurHeadOffset
\setlength\OurHeadOffset{30pt}
\newlength\OurFootOffset
\setlength\OurFootOffset\OurHeadOffset
\addtolength\OurFootOffset{-1em}

%
%Then define the two footers
%
\newcommand{\leftfoot}{\tlap{\llap{\thepage\hspace{0.5em}\blob}\vspace{2em}}}
\newcommand{\rightfoot}{\tlap{\rlap{\blob\hspace{0.5em}\thepage}\vspace{2em}}}

%================================================================================
%Set-up the different page styles used within the document
%================================================================================
\fancypagestyle{FirstPageFrontCover}{\fancyhf{}
  \fancyhead[L]{}
  \fancyfoot[L]{}
  \renewcommand\headrulewidth{0pt}
  \renewcommand\footrulewidth{0pt}
  \renewcommand\thepage{}
  \vspace*{6cm}
  \pagenumbering{alph}
  \renewcommand\thepage{\alph{page}}
}
\fancypagestyle{ContinuationInsideFrontCover}{\fancyhf{}
  \fancyhead[L]{}
  \fancyfoot[L]{}
  \renewcommand\headrulewidth{0pt}
  \renewcommand\footrulewidth{0pt}
  \renewcommand\thepage{}
  \vspace*{4cm}
  \renewcommand\thepage{\alph{page}}
}
\fancypagestyle{FirstPageFrontMatter}{\fancyhf{}%Always page i, so that's a RO page
  \renewcommand\headrulewidth{0pt}
  \renewcommand\footrulewidth{0pt}
  \pagenumbering{roman}
  \renewcommand\thepage{\roman{page}}
}
\fancypagestyle{ContinuationPageFrontMatter}{\fancyhf{}
  \fancyheadoffset\OurHeadOffset% How far to put header and footer outside of text width
  \fancyfootoffset\OurFootOffset% How far to put header and footer outside of text width
  \fancyhead[LO]{}
  \fancyhead[LE, RO]{\large\color{dsiwgAccentColour} {\rotatebox{90}{\llap{\leftmark\hspace{2em}}}}}
  \fancyfoot[LO]{}
  \fancyfoot[LE]{\leftfoot}
  \fancyfoot[RO]{\rightfoot}
  \renewcommand\headrulewidth{0pt}
  \renewcommand\footrulewidth{0pt}
  \renewcommand\thepage{\roman{page}}
}
\fancypagestyle{Standard}{\fancyhf{}
  \fancyheadoffset\OurHeadOffset% How far to put header and footer outside of text width
  \fancyfootoffset\OurFootOffset% How far to put header and footer outside of text width
  \fancyhead[LO]{}
  \fancyhead[LE, RO]{\large\color{dsiwgAccentColour} {\rotatebox{90}{\llap{\leftmark\hspace{2em}}}}}
  \fancyfoot[LO]{}
  \fancyfoot[LE]{\leftfoot}
  \fancyfoot[RO]{\rightfoot}
  \renewcommand\headrulewidth{0pt}
  \renewcommand\footrulewidth{0pt}
  \renewcommand\thepage{\arabic{page}}
}
\fancypagestyle{FirstPageAppendix}{\fancyhf{}
  \fancyhead[LO]{\large\color{dsiwgAccentColour} [Warning: Draw object ignored]}
  \fancyhead[LE]{\large\color{dsiwgAccentColour} [Warning: Draw object ignored]}
  \fancyfoot[LO]{}
  \fancyfoot[LE]{}
  \renewcommand\headrulewidth{0pt}
  \renewcommand\footrulewidth{0pt}
  \renewcommand\thepage{\arabic{page}}
}
\fancypagestyle{FirstPageTitle}{\fancyhf{}
  \fancyhead[L]{\large\color{dsiwgAccentColour} [Warning: Draw object ignored]}
  \fancyfoot[L]{\large{\scriptsize\textcolor[rgb]{0.5019608,0.0,0.0}{MISRA internal use only}}\hfill \thepage{}}
  \renewcommand\headrulewidth{0pt}
  \renewcommand\footrulewidth{0pt}
  \renewcommand\thepage{\roman{page}}
}
\fancypagestyle{FirstPage}{\fancyhf{}
  \fancyfootoffset\OurFootOffset% How far to put header and footer outside of text width
  \fancyhead[LO,LE,RO,RE]{}
  \fancyfoot[LO,RE]{}
  \fancyfoot[LE]{\leftfoot}
  \fancyfoot[RO]{\rightfoot}
  \renewcommand\headrulewidth{0pt}
  \renewcommand\footrulewidth{0pt}}

\setlength\tabcolsep{1mm}
\renewcommand\arraystretch{1.3}
\sloppy

%================================================================================
%Utility to make it easier for us to make "section X" a hyperlink, rather than just the X
% Needed because \autoref is inconsistent in references appendices -
% Some were shown as "section M", others as "Appendix O".
% Also useful for referencing subsubsections.
%================================================================================

\newcommand{\dsiwgRef}[2]{\hyperref[#2]{#1}~\ref{#2}}

%================================================================================
% Utility to make it easier to add entries to the acronym list
%================================================================================
\newcommand{\acrentry}[1]{\acrshort{#1}&\acrlong{#1}\\\hline}

%================================================================================
%Prevent the bibliography macros from inserting their own ``References'' heading.
%================================================================================
\patchcmd{\thebibliography}{\section*{\refname}}{}{}{}

%================================================================================
%Set-up the list of Abbreviations and Glossary
%================================================================================
% TGR moved to end
% \loadglsentries{glossary}
% Next line commented out as glossaries not yet generated by package glossaries
%\makeglossaries

%================================================================================
%Allow searching of ligatures in Adobe Reader
%================================================================================
\input{glyphtounicode}

\pdfglyphtounicode{f_f}{FB00}
\pdfglyphtounicode{f_f_i}{FB03}
\pdfglyphtounicode{f_f_l}{FB04}
\pdfglyphtounicode{f_i}{FB01}

\pdfgentounicode=1
%
% Define the tick mark for use in tables 
%
\newcommand\tick{\ding{51}}
%
% Make table and figure captions use section-counter format
%
%\usepackage{chngcntr}
%\counterwithin{figure}{section}

\usepackage[
	style=listdotted,%
	section=subsection,%
	numberedsection=autolabel,%
	nonumberlist,%
	hyperfirst=false,% doesn't work, for some reason.
	acronym,%
	toc
	]{glossaries-extra}
	% hyperfirst=false suppresses all first-reference hyperlinks, but we want hyperlinks for all main glossary entries.
	% This sets them as if they've already been referenced.
	\glsunsetall[main]
\setabbreviationstyle[acronym]{long-short}
\newglossary*{definitions}{Normative Definitions}
\makenoidxglossaries
\loadglsentries{glossary}
