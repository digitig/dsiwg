%================================================================================
%       Safety Critical Systems Club - Data Safety Initiative Working Group
%================================================================================
%                       DDDD    SSSS  IIIII  W   W   GGGG
%                       D   D  S        I    W   W  G   
%                       D   D   SSS     I    W W W  G  GG
%                       D   D      S    I    WW WW  G   G
%                       DDDD   SSSS   IIIII  W   W   GGG
%================================================================================
%               Data Safety Guidance Document - LaTeX Source File
%================================================================================
%
% Description:
%   Section regarding COVID'19, also known as the CCP Virus.
%
%================================================================================
\section{Covid-19\index{Covid-19|textbf} (Informative)} \label{bkm:Covid19}
%\dsiwgSectionQuote{Common sense tells us that the government's attempts to solve large problems more often create new ones. Common sense also tells us that a top-down, one-size-fits-all plan will not improve the workings of a nationwide health-care system that accounts for one-sixth of our economy}{ Sarah Palin}

%\dsiwgSectionQuote{When we make decisions in our personal and professional lives, we typically start with some form of data. The very word ‘data’ derives from the Latin meaning ‘something given’. But who gave it? Where is it from? Should I accept it at face value?}{Adrian Smith CEO of the Alan Turing Institute}

\dsiwgSectionQuote{The public health community wants a safe and effective [COVID-19] vaccine as much as anybody could want it. But the data have to be clear and compelling.}{Michael Osterholm}

\subsection{Covid-19 and Data}
The Covid 19 crisis has highlighted a number of areas where better data and data management could have improved outcomes and hence reduced the death toll. Such pandemic-related data is therefore very much safety-related data. 

Some issues related to Covid-19 data are:
\begin{enumerate}
\item The lack of \index{Consistency!Property Handling}\gls{consistency} and standardisation in handling the data, e.g:
  \begin{enumerate}[label=\color{dsiwgAccentColour}{\alph*.}]
  \item In predictive models\index{Predictive Models} where many assumptions may be wrong, or the algorithms inappropriate
  \item Presentation of statistics in a selective or misleading way
  \item Methods of data collection which may be selective or incomplete. For instance reporting on the number of positive test cases is always misleading, as many people may be asymptomatic with the virus so never get tested.
  \item Calculations and filtering
  \item Allowances for delays in collection or processing
  \item Intentional and unintentional bias
  \item Use of averaging (e.g.\ moving averages) and smoothing of plots hiding sudden increases
  \item Loss of data (e.g.\ in the recent UK Test and Trace system due to old versions of Excel\index{Excel})
  \end{enumerate}
All of which have prevented any meaningful comparisons internationally, even between countries in Western Europe. They also lead to public confusion; this in turn leads to mistrust and a refusal to abide by guidance and regulations. 

\item The poor data within the Test and Trace systems is a major factor in the failure of these systems. If data is not accurate, timely or complete then contacts cannot be traced in time, and the effort put into the activity is wasted. These systems in the UK need huge improvement as there is currently low contact performance and hence very poor outcomes. Background reading on this may be found at this link: \href{https://www.bbc.co.uk/news/health-55008133}{https://www.bbc.co.uk/news/health-55008133}
\end{enumerate}

\subsection{Systems Involved with Covid-19 Data}
Many systems have been created or re-deployed to help manage the pandemic. These systems consume and produce vast amounts of data, some of which is critical and could affect safety of individuals or the general population. Some identified systems are shown in \autoref{tab:PandemicSystems}. For each of these, it is worth thinking through some basic data failure modes, e.g.\ data is lost, late, incorrect or incomplete. For instance if we are running an infection / spread model and we feed it with stale data, then its predictions will clearly be inaccurate.
\begin{longtable}{|L{\dsiwgColumnWidth{0.25}}|L{\dsiwgColumnWidth{0.25}}|L{\dsiwgColumnWidth{0.25}}|L{\dsiwgColumnWidth{0.25}}|}
  \caption{Systems Involving Data Used to Manage the Pandemic}
  \label{tab:PandemicSystems}
  \\\hline
  \endfirsthead
  \caption[]{Systems Involving Data Used to Manage the Pandemic (continued)}
  \\\hline
  \endhead
  \multicolumn{4}{r}{\sl Continued on next page}
  \endfoot\endlastfoot

  Analysis of air flow and particles & 
  Satellite imagery (Wuhan) &
  Video conferencing &
  Risk assessment systems\\\hline
  Infection / spread models (inc new variants) &
  Itinerary systems &
  Remote consultation systems &
  Computational bioinformatics tools\\\hline
  Infra-red / thermal cameras &
  Infection testing systems &
  Ventilators / other patient management devices &
  Appointment systems\\\hline
  Track and Trace apps &
  Antibody testing systems &
  Personal risk profiling apps / systems &
  Border Control / Quarantine systems\\\hline
  Track and Trace back office systems &
  Drug trials systems / data &
  Allocation / reservation / booking systems &
  Risk profiling / prioritising for vaccination\\\hline
  Track and Trace service &
  Ventilation models \& UV sterilisers &
  Models of built environments &
  Digital Twins (systems and biological: lungs, etc.)\\\hline
  Supply chain systems &
  Behavioural models &
  Safety analyses (STAMP/STPA), etc. &
  (Automatic) cleaning systems\\\hline
  Virus aerosols modelling &
  Analysis of delays in system of reporting / actions &
  Modelling / public perception of the disease &
  Virus shedding models\\\hline
%
  Vaccination booking / tracking / monitoring &
  Sanitiser systems &
  Lockdown easing models &
  Vaccination Passports
  \\\hline
%
  Vaccination production data &
  Vaccination trials and reporting data &
  Vaccination ``Yellow Card'' &
  Cross-system data sharing
  \\\hline
%
  PPE testing results &
  Data used to inform public perceptions &
  No coordination across international boundaries --- incompatible systems &
  Use of blockchain to validate Covid and vaccination status\\
  \hline
\end{longtable}

\subsection{Falsification/Misinformation of Data}
One serious and perhaps unexpected aspect of the pandemic is that of misinformation. There are people either in denial of the virus’s dangers, refusing to socially distance or refusing vaccinations. Reasons for these behaviours have generally been driven by intentional misinformation, ignorance, superstition, or economics. Marianna Spring, the BBC’s specialist reporter covering disinformation and social media put the problem of misinformation succinctly when she stated “The problem with misinformation is that it is popular.” See Barack Obama: One election won't stop US `truth decay' --- BBC News at \href{https://www.bbc.co.uk/news/election-us-2020-54910344}{https://www.bbc.co.uk/news/election-us-2020-54910344}. Methods need to be devised or improved to prevent this effect, and to restore trust in carefully managed data. 

\subsection{Rumsfeld’s\index{Rumsfeld, Donald} known unknown\index{Known Unknowns} and unknown unknown\index{Unknown Unknowns} data conundrum}

It is clear that until China reported to the world that Covid 19 had emerged as a threat, that data about the virus was an unknown unknown. However, we know there are thousands of viruses in animals that could pose a threat. These all need analysis and it may be possible to use massive computer analysis of genetic data to identify likely new threats. 
	
A paper was
given by Nick Hales as part of the 2021 Safety-Critical Systems Symposium which gives more detail on this topic {\it“Data Safety in Virus Outbreaks --- Lessons learnt and Recommendations”}~\cite{citation:SCSC161}.

\subsection{Learning}

It is important to learn from these deficiencies because, while Covid 19 has brought tragedies with it, it is unlikely to be the last, or indeed, the most dangerous virus we will face. We must do better next time.
