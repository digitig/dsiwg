%================================================================================
%       Safety Critical Systems Club - Data Safety Initiative Working Group
%================================================================================
%                       DDDD    SSSS  IIIII  W   W   GGGG
%                       D   D  S        I    W   W  G   
%                       D   D   SSS     I    W W W  G  GG
%                       D   D      S    I    WW WW  G   G
%                       DDDD   SSSS   IIIII  W   W   GGG
%================================================================================
%               Data Safety Guidance Document - LaTeX Source File
%================================================================================
%
% Description:
%   Glossary file defining all of the terms/acrnoyms used within the guidance
%   document.
%
%================================================================================

%
%This is very much a work-in-progress and not all entries are present. But, like
%the references using BibTeX, if used properly LaTeX will manage all of our
%acronyms and glossary entries for us, if we choose to go down that route.
%This is merely highlighting what is possible for now.
%

%A
\newacronym{aal}{AAL}{above aerodrome level}
\newacronym{adiru}{ADIRU}{air data inertial reference unit}
\newacronym{ai}{AI}{artificial intelligence}
\newacronym{aoa}{AoA}{angle of attack}
\newacronym{arp}{ARP}{aerospace recommended practice}
\newacronym{arq}{ARQ}{automatic repeat-request}
\newacronym{atm}{ATM}{air traffic management}
\newacronym{atsb}{ATSB}{Australian Transport Safety Bureau}
%B
\newacronym{bit}{BIT}{built in test}
\newacronym{bite}{BITE}{built in test equipment}
%C
\newacronym{cfit}{CFIT}{controlled flight into terrain}
\newacronym{cns}{CNS}{communications, navigation, and surveillance}
\newacronym{cod}{CoD}{certificate of design}
\newacronym{cots}{COTS}{commercial off-the-shelf}
\newacronym{cpu}{CPU}{central processing unit}
\newacronym{csv}{CSV}{comma separated variable}
\newacronym{ct}{CT}{computed tomography}
%D
\newacronym{dcb}{DCB}{data coordination board}
\newacronym{dme}{DME}{distance measuring equipment}
\newacronym{doc}{DoC}{Department of Corrections}
\newacronym{dracas}{DRACAS}{defect reporting and corrective action system}
\newacronym{dsal}{DSAL}{data safety assurance level}
\newacronym{dsiwg}{DSIWG}{Data Safety Initiative Working Group}
\newacronym{dsg}{DSG}{data safety guidance}
\newacronym{dsmp}{DSMP}{data safety management plan}
%E
\newacronym{ecs}{ECS}{electronic chart system}
\newacronym{ecu}{ECU}{electronic control unit}
\newacronym{edm}{EDM}{entry demonstrator module}
\newacronym{egpws}{EGPWS}{enhanced ground proximity warning system}
\newacronym{ehr}{EHR}{electronic health record}
\newacronym{enc}{ENC}{electronic navigational chart}
\newacronym{esa}{ESA}{European Space Agency}
%F
\newacronym{faa}{FAA}{Federal Aviation Administration}
\newacronym{fcpc}{FCPC}{flight control primary computer}
\newacronym{fdal}{FDAL}{functional design assurance level}
\newacronym{fmgs}{FMGS}{flight management guidance system}
\newacronym{fms}{FMS}{flight management system}
%G
\newacronym{gcs}{GCS}{ground control station}
\newacronym{gp}{GP}{general practitioner}
\newacronym{gps}{GPS}{Global Positioning System}
\newacronym{gtols}{GTOLS}{\glsfmtshort{gps} take-off and landing system}
%H
\newacronym{hse}{HSE}{Health and Safety Executive}
\newacronym{hazop}{HAZOP}{hazard and operability study}
\newacronym{hums}{HUMS}{health and usage monitoring system}
%I
\newacronym{ibm}{IBM}{International Business Machines Corporation}
\newacronym{icao}{ICAO}{International Civil Aviation Organization}
\newacronym{icd}{ICD}{interface control document}
\newacronym{idal}{IDAL}{item development assurance level}
\newacronym{imc}{IMC}{instrument meteorological conditions}
\newacronym{imu}{IMU}{inertial measurement unit}
\newacronym{ip}{IP}{internet protocol}
\newacronym{irs}{IRS}{inertial reference system}
\newacronym{iso}{ISO}{International Standards Organization}
\newacronym{iv}{IV}{intravenous}
%J
%K
%L
\newacronym{lhr}{LHR}{London Heathrow airport}
\newacronym{llm}{LLM}{large language model}
%M
\newacronym{maib}{MAIB}{Maritime Accident Investigation Board (\textit{or} Branch)}
\newacronym{mca}{MCA}{Maritime and Coastguard Agency}
\newacronym{mcas}{MCAS}{manoevring characteristics augmentation system}
\newacronym{ml}{ML}{machine learning}
\newacronym{msaw}{MSAW}{minimum safe altitude warning}
%N
\newacronym{nan}{NaN}{not a number}
\newacronym{notam}{NOTAM}{notice to airmen}
%O
\newacronym{odr}{ODR}{organizational data risk}
\newacronym{oow}{OOW}{officer of the watch}
\newacronym{osi}{OSI}{open system interconnection}
%P
\newacronym{pcr}{PCR}{polymerase chain reaction}
%Q
%R
\newacronym{rda}{RDA}{radar Doppler altimeter}
%S
\newacronym{saiwg}{SAIWG}{Safe \gls{ai} Working Group}
\newacronym{sar}{SAR}{search and rescue}
\newacronym{saswg}{SASWG}{Safety of Autonomous Systems Working Group}
\newacronym{scsc}{SCSC}{Safety-Critical Systems Club}
\newacronym{sil}{SIL}{safety integrity level}
\newacronym{smp}{SMP}{safety management plan}
\newacronym{sop}{SOP}{standard operating procedure}
\newacronym{sotif}{SOTIF}{safety of the intended functionality}
\newacronym{sss}{SSS}{Safety-critical Systems Symposium}
%T
\newacronym{tcas}{TCAS}{traffic collision avoidance system}
%U
\newacronym{uas}{UAS}{unmanned air system}
\newacronym{usb}{USB}{universal serial bus}
%V
\newacronym{vhf}{VHF}{Very High Frequency}
\newacronym{vor}{VOR}{\glsxtrshort{vhf} omnidirectional range}
\newacronym{vmc}{VMC}{visual meteorological conditions}
\newacronym{vms}{VMS}{voyage management system}
%W
%X
\newacronym{xml}{XML}{extensible markup language}
%Y
%Z

\longnewglossaryentry{accuracy}{%
	name={accuracy},%
	plural={accuracies}}%
	{Closeness of agreement between a test result and the accepted reference value. Note that a test result can be observations or measurements. ISO\ 19113:2005 \cite{citation:ISO19113}

	A degree of conformance between the estimated or measured value and the true value. (EU) No 73/2010 \cite{citation:EU732010}

	(Temporal)  Correctness of the temporal references of an item (reporting of error in time measurement). Correctness of ordered events or sequences, if reported. Validity of data with respect to time. ISO\ 19138:2006 \cite{citation:ISO19138}}
% Data Artefact is also in the normative definitions, and should not be repeated.
%\newglossaryentry{artefact}{name={(data) artefact},description={ An item, or collection of items, that provides a useful perspective on data
%		generated, processed or consumed by a system.}}
% I've split data assurance level and software assurance level
\newglossaryentry{data assurance level}{name={data assurance level},description={The required \index{Assurance Level}assurance level for the aeronautical data process is identified, based on the overall system architecture through allocation of risk determined using a preliminary \index{Safety Assessment!System}system safety assessment. RTCA/DO-200A \cite{citation:ED76}}}

\longnewglossaryentry{adaptation data}{%
	name={adaptation data},%
	plural={adaptation data}}%
	{Data used to customise elements of the \gls{atm} System for their designated purpose.
		Adaptation data\index{Adaptation Data} is used to customise elements of the \gls{cns} / \gls{atm} system for its designated purpose at a specific location.
		These systems are often configured to accommodate site-specific characteristics.
		These site dependencies are developed into sets of adaptation data\index{Adaptation Data}.
		Adaptation data\index{Adaptation Data} includes data that configures the software for a given geographical site, and data that configures a workstation to the preferences and / or functions of an operator.
		Examples include, but are not limited to:
		\begin{enumerate}
			\item Geographical Data - latitude and longitude of a radar site.
			\item Environmental Data - operator selectable data to provide their specific preferences.
			\item Airspace Data - sector-specific data.
			\item Procedures - operational customisation to provide the desired operational role.
		\end{enumerate}
		Adaptation data\index{Adaptation Data} may take the form of changes to either database parameters or take the form of pre-programmed options.
		In some cases, adaptation data\index{Adaptation Data} involves re-linking the code to include different libraries.
		Note that this should not be confused with recompilation in which a completely new version of the code is generated. ED-153 \cite{citation:ED153}}

\longnewglossaryentry{aeronautical data}{name={aeronautical data}, plural={aeronautical data}}%
{A representation of aeronautical facts, concepts or instructions in a formalized manner suitable for communication, interpretation or processing. (EU) No 73/2010 \cite{citation:EU732010}

Data used for aeronautical applications such as navigation, flight planning, flight simulators, terrain awareness, and other purposes. RTCA/DO-178C \cite{citation:ED12C}}

\newglossaryentry{availability}{name={availability},plural={availabilities},description={The property of being accessible and usable upon demand by an authorized entity. ISO\ 27001:2013 \cite{citation:ISO27001:2013}}}

\newglossaryentry{completeness}{name={completeness},description={Completeness of the data provided. RTCA/DO-200A}}

\newglossaryentry{software assurance level}{name={assurance level, software},text={software assurance level},description={An indication of how much assurance is required (commensurate to risk) before deploying software into an operational system. J Spriggs, based on (EC) No 482/2008 \cite{citation:EC4822008}}}

\longnewglossaryentry*{configuration data}{name={configuration data}, plural={configuration data}}%
{Data that configures a generic software system to a particular instance of its use. (EC) No 482/2008 \cite{citation:EC4822008}

Data that configures a generic software system to a particular instance of its use (e.g., data for flight data processing system for a particular airspace, by setting the positions of airways, reporting points, navigation aids, airports and other elements important to air navigation). ED-153 \cite{citation:ED153}Data that configures a generic software system to a particular instance of its use (e.g., data for flight data processing system for a particular airspace, by setting the positions of airways, reporting points, navigation aids, airports and other elements important to air navigation). ED-153 \cite{citation:ED153}}

\newglossaryentry{confidentiality}{name={confidentiality},description={The property that information is not made available or disclosed to unauthorized individuals, entities, or processes. ISO27001:2013 \cite{citation:ISO27001:2013}}}

\newglossaryentry{consistency}{name={consistency (data)},text={consistency},description={The property that the data adheres to a common world view (e.g., units).} }

\newglossaryentry{continuity}{name={continuity (data)},text={continuity},description={The property that the data is continuous and regular without gaps or breaks.}}

\newglossaryentry{correctness}{name={correctness (data)},text={correctness},description={\index{Consistency!Self}self-\gls{consistency}, protection against alteration or corruption and \index{Consistency!With Requirements}\gls{consistency} with the functional requirements of the \gls{data-driven system}. IEC 61508 Part 3 \cite{citation:iec615083}}}

\newglossaryentry{coupling}{name={coupling (data)}, text={coupling},description={The dependence of a software component on data not exclusively under the control of that software component. RTCA/DO-178C \cite{citation:ED12C}}} %doesn't seem to appear in text.

\newglossaryentry{criticality}{name={criticality (data)},text={criticality},description={Classification of data by the potential effect of erroneous data on the expected operation that is supported by that data. RTCA/DO-200A \cite{citation:ED76}}}

\newglossaryentry{critical data}{name={critical data},description={Data with an \index{Integrity Property}integrity level as defined in Chapter 3, Section 3.2 point 3.2.8(a) of Annex 15 to the Chicago Convention, i.e., \index{Integrity Property}integrity level one in one hundred million: there is a high probability when using corrupted critical data that the continued safe flight and landing of an aircraft would be severely at risk with the potential for catastrophe. (EU) No 73/2010 \cite{citation:EU732010}}}

\newglossaryentry{customisation data}{name={customisation data},description={Data used to configure a system or component. Def(Aust)5679 \cite{citation:DEFOz}}}

\longnewglossaryentry*{data}{name={data},plural={data}}{A thing given or granted; something known or assumed as fact, and made the basis of reasoning or calculation; an assumption or premiss from which inferences are drawn. Oxford English Dictionary (OED)
	
A reinterpretable representation of information in a formalized manner suitable for communication, interpretation or processing. ISO/IEC\ 2382 \cite{citation:ISO23821}}

\newglossaryentry{data consistency}{name={data consistency},description={See \glsname{consistency}}}

\newglossaryentry{data continuity}{name={data continuity},description={See \glsname{continuity}}}

\newglossaryentry{data correctness}{name={data correctness},description={See \glsname{correctness}}}

\newglossaryentry{data criticality}{name={data critiality},description={See \gls{criticality}}}

\newglossaryentry{data customisation}{name={data customisation},description={See \gls{customisation}}}
	
\newglossaryentry{database}{name={database},description={A set of data, part or the whole of another set of data, consisting of at least one file that is sufficient for a given purpose or for a given data processing system. RTCA/DO-178C}}

\longnewglossaryentry{data chain}{name={data chain}}{An `Aeronautical Data Chain' is a conceptual representation of the path that a set, or element of aeronautical data takes from its creation to its end use. An aeronautical data chain is a series of interrelated links wherein each link provides a function that facilitates the origination, transmission and use of aeronautical data for a specific purpose. RTCA/DO-200A \cite{citation:ED76}

A collection of organizational data processing functions, where data is transferred from one chain participant to another between data origination and end use. P. Ensor \cite{citation:Ensor2009}

Any combination of two or more data elements, data items, data codes, and data abbreviations in a prescribed sequence to yield meaningful information; for example, ``date'' consists of data elements year, month, and day. McGraw-Hill Dictionary \cite{citation:McGrawHill}}

\newglossaryentry{data dictionary}{name={data dictionary},plural={data dictionaries},description={The detailed description of data, parameters, variables, and constants used by the system. RTCA/DO-178C}}

\newglossaryentry{data-driven system}{name={data-driven system},description={System which relies upon \gls{configuration data} or lookup tables to define the functionality of the system. IEC 61508 Part 4 \cite{citation:iec615084}}}

\newglossaryentry{data-intensive system}{name={data-intensive system},description={Systems which make extensive use of large amounts of data. & N. Storey \cite{citation:StoreyFaulkner20031}}}

