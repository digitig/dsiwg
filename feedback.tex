%================================================================================
%       Safety Critical Systems Club - Data Safety Initiative Working Group
%================================================================================
%                       DDDD    SSSS  IIIII  W   W   GGGG
%                       D   D  S        I    W   W  G   
%                       D   D   SSS     I    W W W  G  GG
%                       D   D      S    I    WW WW  G   G
%                       DDDD   SSSS   IIIII  W   W   GGG
%================================================================================
%               Data Safety Guidance Document - LaTeX Source File
%================================================================================
%
% Description:
%   feedback section.
%   Records feedback from users and actions taken
%================================================================================
\section{Feedback (Discursive)} \label{bkm:feedback}

\dsiwgSectionQuote{I think it’s very important to have a feedback loop, where you’re constantly thinking about what you’ve done and how you could be doing it better.}{Elon Musk}

Comments from users of the Guidance document are always welcome.
\todo{This section to be omitted from published document}
This section records feedback received and the actions we have taken as a result.

Fully completed entries have been ``greyed out''. Green background has been used for comments addressed in the current update.
Those that are not yet fully addressed retain a white background.
\subsection{Submitted by Mark Templeton}
\subsubsection{Main comments on the Guidance in general and which became apparent during the Data Safety course}

\begin{longtable}[H]
{|L{\dsiwgColumnWidth{0.08}}|L{\dsiwgColumnWidth{0.46}}|L{\dsiwgColumnWidth{0.46}}|}
\hline
\TableHeadColour{Num} & \TableHeadColour{Issue} & \TableHeadColour{Action}\\
\hline
  \endfirsthead
  \hline\TableHeadColour{Num} & \TableHeadColour{Issue} & \TableHeadColour{Action}\\\hline
  \endhead
  \endfoot
  \endlastfoot

1&A process flow diagram is required early in the document, to provide an Idiot's Guide to the process.&**TBD** Deferred to next full update.\\\hline
2&The Organisational Data Risk Assessment does not fit well into the overall DSAL-based approach. We need to find a way to be clearer how this fits into the overall process.&**TBD** This requires significant change to the order of the document, and could not be addressed whilst maintaining the section numbering from version 3.0.\\\hline
%
\rowcolor{gray}
3&
Section 6.3.1, para following first table: Explains that it's the *lowest* likelihood which should be used in selecting the overall DSAL. This is very unusual, and needs to be expanded upon.
&Explanatory text has been added to 6.3.1.\\\hline
%
\rowcolor{gray}
4&Section 6.4.2.2: The use of the tables in 6.4.2.3 to 6.4.2.10 is not explained and currently has to be worked out from the later "worked example". An explanation of how to use the tables in sections 6.4.2.3 to 6.4.2.10 is needed, to explain how they should be applied, including:
    \begin{itemize}
        \item The pattern matching,
        \item The meaning of the "dots",
        \item Why we keep the dots.
    \end{itemize}&A step-by-step guide has been added to 6.4.2.2 which addresses each of these issues.\\
  \hline
\end{longtable}

\clearpage
\subsubsection{General comments from Data Safety course}

\begin{longtable}[H]
{|L{\dsiwgColumnWidth{0.08}}|L{\dsiwgColumnWidth{0.46}}|L{\dsiwgColumnWidth{0.46}}|}
\hline
\TableHeadColour{Num} & \TableHeadColour{Issue} & \TableHeadColour{Action}\\
\hline
  \endfirsthead
  \hline\TableHeadColour{Num} & \TableHeadColour{Issue} & \TableHeadColour{Action}\\\hline
  \endhead
  \endfoot
  \endlastfoot
  
  \rowcolor{gray}
  1&The Creative Commons Attribution 4.0 licence has been adopted, but what is its significance? How does it differ from Gnu?&No action.\\\hline
  %
  \rowcolor{gray}
  2&Section 5.2.2.1: "a list of anecdotal "ways that data can cause problems" is also available". From where?&Reworded to refer to Appendix H and to suggest that each domain should maintain its own historical list.\\\hline
  %
  \rowcolor{gray}
  3&Section 5.2.2.4: "Plan) requires updating" should read "Plan) requires review to determine if it needs updating".&Done\\\hline
  \rowcolor{gray}
  4&Section 5.2.3: Para 3 ends "HAZOP Guidewords have been determined". Forward reference to that section should be given.&Done\\\hline
  %
  5&Section 5.3.2.1 includes a DSAL table. However the table does not make sense at this point in the document, as the definition of each likelihood category is not given until page 28, twelve pages later. Need to bring definitions forward, create a reference, or move both table and definitions into the earlier section on the DSMP.&New paragraph added under table as a short-term fix. **Long term fix requires reordering of the document**\\\hline
  %
  \rowcolor{gray}
  6&Section 5.3.2.2: Para 2suggests that assurance levels may come "from perspectives other than data safety". The paragraph should end with a note to suggest that "Such mappings should be included within the DSMP".&Done, with the text ``Where such an approach is used, the mapping to \glspl{dsal} should be included within the \gls{dsmp}''.\\\hline
  %
  \rowcolor{gray}
  7&Section 5.4.2: Treatment "Transfer": Transfer can be a very risky strategy and in military aviation may be prohibited. It should not just be "accepted" by both parties, but needs to be "agreed" by both parties.&Done\\\hline
  %
  \rowcolor{gray}
  8&Section 5.4.3: Second para is a re-wording of 5.3.3 first para.&Addressed by cross-referncing 5.3.3 from 5.3.4\\\hline
  %
  9&Section 6.1.3: Talks about the Data Safety Culture Questionnaire and points to Appendix C. However little guidance on interpreting the results is provided.&**TBD** Needs input from author of this section.\\\hline
  %
  \rowcolor{gray}
  10&Section 6.1.6 table: Items 10 and 12 appear to have significant overlap and caused some confusion. In the case of an aircraft system, is \#10 intended to cover configuration for an F16 aircraft, and \#12 for a specific F16, or does \#10 set up the general menus and \#12 configure menus specific to F16, C-17, etc? We were able to interpret \#11 as "to get it to do what you want", but considering real cases against \#10 and \#12 caused the confusion and made us realise that we were uncertain about the boundaries between all three.&Already described in Appendix E. Extra words added to 6.1.6.\\\hline
  %
  \rowcolor{gray}
  11&Section 6.2.1: Para 2 introduces data properties. A pointer to 6.2.3 (where they are defined) would be helpful.&Sentence reworded and cross reference inserted\\\hline
  %
  \rowcolor{gray}
  12&Section 6.2.3 table: Item "Intended Destination / Usage": Replace "have it" with "have access to it" to make the property more generic.&Done\\\hline
  %
  \rowcolor{gray}
  13&Section 6.2.4: Should explain that the guidewords are to be used against the list of properties which were already defined in section 6.2.3. It would also be helpful to reference out to an explanation of HAZOP.&Done. Also added ref to (withdrawn) 00-58.\\\hline
  %
  \rowcolor{gray}
  14&Section 6.3.1, para following first table: "DSAL is the lowest" should be "DSAL is associated with the lowest".&Done.\\\hline
  %
  \rowcolor{gray}
  15&Section 6.4.2, para 3: Typo: A "methodology" is a catalogue or study of methods. The word "method" should be used in the Guidance, in preference to methodology.&Done. Also fixed in Appendix D.\\\hline
  %
  \rowcolor{gray}
  16&Section 6.4.2.1 table: Entry "Editing limitations" used the domain-specific term "encapsulation of data". Further explanation would help.&Added footnote.\\\hline
  %
  %TGR Start
  \rowcolor{gray}
17&Section 6.4.2.2 table 1: Five new data categories are listed with no explanation of where they have magically appeared from.&They do align now (one is a subset of the other).\\\hline
%TGR End
18&Section 6.4.2.2 table 2: This table of "data properties" is the same as the list of Data Properties presented in section 6.2.3. If we must list them twice, at least cross reference the two tables, so that the reader of section 6.4.2.2 does not wonder how these also magically appear. It might actually be better to embed the Abbreviations in the earlier table, to avoid the need for this one along with cross-referencing.&Cross reference added. **Long term fix: merge tables**\\\hline
19&Section 6.4.2.2 bulleted list following table 2: Why do we have a new set of headings, instead of the list of data type "headings" within the table of section 6.1.6? There appears to be no good reason for introducing a new list and we probably need to harmonise the two lists.&**TBD**\\\hline
20&Section 6.4.2.4: Title is "Data Design" yet only the first four entries fit this title. A better name may be "Construction of the data storage structures and methods".&Inserted descriptive text. **Long term fix: retitle**\\\hline
% TGR Start
\rowcolor{gray}
21&Section 6.4.2.5: Is called "Data Implementation" but is really managing or checking the values themselves.&This table and next restructured and amended.\\\hline
\rowcolor{gray}
22&Section 6.4.2.6: Is called "Data Migration" but it is necessary to read the contents of the table before you can deduce that this means the migration from one implementation to another, as opposed to the movement or gradual corruption of data.&This table and previous restructured and amended.\\\hline
%TGR End
23&Section 6.4.2.6 and 6.4.2.7 both include "Client sign-off" however there is no indication of how this maintains a property of the data.&**TBD**\\\hline
24&Section 6.4.2.7: "Client sign-Off of Data": How can the client determine that the data is "appropriate"?&**TBD**\\\hline
25&Section 6.4.2.8: Seems very incomplete: What about stress testing, limits of range, failure values, ensuring that the whole number space is tested?&**TBD**\\\hline
26&Section 6.4.2.8: The entries "Using Testbed" and "Using Dedicated Platform" appear to be almost the same, and "Using Testbed" seems to be a subset of "Using Dedicated Platform".&**TBD**\\\hline
27&Section 6.4.2.8: Again, we have "Client Sign-Off" as "appropriate" with no indication of how this can affect a data property.&**TBD**\\\hline
%
\rowcolor{gray}
28&Section 7.2.2 needs a reference to Appendix B, page 61, to make it easier to interpret this section.&Done.\\\hline
%
\rowcolor{gray}
29&Section 7.2.2, Q8: First para "have been" should read "need to be validated".&Inserted text.\\\hline
%
\rowcolor{gray}
30&Section 7.2.2, para prior to second Q2: Typo: Assessments => Assessment.&Done.\\\hline
%
\rowcolor{gray}
31&Section 7.2.2, third para after second Q5: Typo: "Health Organisations" => "Health Organisation"&Done.\\\hline
%
\rowcolor{gray}
32&Section 7.3: Second bulleted list: I know the list is only an example, but it may be prudent to add "Will existing barriers be lost as a consequence of the new system?" This is an aspect that we don't seem to talk about anywhere.&Done.\\\hline
%
\rowcolor{gray}
33&Section 7.5 para prior to first table: might be clearer to append "and are relevant to verification data".&Done.\\\hline
%
\rowcolor{gray}
34&Section 7.5: It would be helpful to state where each table comes from: "System Design, pages 34-35...Data Design, pages 36-37..." Otherwise the section is hard to read, and the empty tables do not make sense.&Added links in header of each table\\\hline
%
\rowcolor{gray}
35&Section 7.5, table 3: I note that Data Quality Trend Analysis has been excluded (as it was a "-" for DSAL 1, not an "R") yet appears to be a particularly relevant technique. This suggests that in the explanation of how to use the tables, we need to suggest that even when a "-" appearing against a given DSAL would be grounds for exclusion, it is worth considering whether domain-specific factors could make the approach worth considering.&Incorporated into the step-by-step explanation of how to use the Technique tables.\\\hline
%
\rowcolor{gray}
36&Section 7.5, table 8 (MEx): Why has "Multiple copies" been excluded?&No action: It's DSAL 1, so not needed.\\\hline
%
37&Section 7.5, final table: Claims "There is no contracted client..."yet earlier we're told about how involved the Health Organisation has been - clearly a prospective client "who are procuring a solution". For each of DD4, TD5 and TD6 there are aspects which should be resolved with the client. What does client sign-off mean? Is it an acceptable approach? If not, do additional techniques need to be applied?&**TBD**\\\hline
%
\rowcolor{gray}
38&Section 7.5, final table: Typo in row E2: developed => development&Done.\\\hline
%
39&Appendix E, section "Configuration", entry "Infrastructure" (row 10): "network addresses, passwords" would seem to sit better under "Adaptation" (row 12).&**TBD**\\\hline
%
\rowcolor{gray}
40&Appendix K quote should probably state "Misquoted from Sun Tzu".&Done.\\
      \hline
\end{longtable}
%
\subsubsection{Other errors found during review}
%
\begin{longtable}[H]
{|L{\dsiwgColumnWidth{0.08}}|L{\dsiwgColumnWidth{0.46}}|L{\dsiwgColumnWidth{0.46}}|}
\hline
\TableHeadColour{Num} & \TableHeadColour{Issue} & \TableHeadColour{Action}\\
\hline
  \endfirsthead
  \hline\TableHeadColour{Num} & \TableHeadColour{Issue} & \TableHeadColour{Action}\\\hline
  \endhead
  \endfoot
  \endlastfoot

  %
  \rowcolor{gray}
  1&Table at the beginning of Appendix H had an underfull hbox, due to column 2 being too narrow.&MDT: Tiny adjustment by taking 0.01 from column 4 and adding it to column 2 made entry H.4 "Interception of Communications" look right.\\\hline
  %
  \rowcolor{gray}
2&Tables and figures were hard to reference, yet references are essential in the generation of claims to comply with the Guidance.&Captions have been added to all tables and figures which may require referencing. In the appendices some tables were not captioned, where it would not be normal practice -- for example the list of abbreviations.\\\hline
\end{longtable}

\clearpage
\subsection{Submitted by Dave Banham}
\begin{longtable}[H]
{|L{\dsiwgColumnWidth{0.08}}|L{\dsiwgColumnWidth{0.46}}|L{\dsiwgColumnWidth{0.46}}|}
\hline
\TableHeadColour{Num} & \TableHeadColour{Issue} & \TableHeadColour{Action}\\
\hline
  \endfirsthead
  \hline\TableHeadColour{Num} & \TableHeadColour{Issue} & \TableHeadColour{Action}\\\hline
  \endhead
  \endfoot
  \endlastfoot

  %
  \rowcolor{gray}
  1&DSG v3.0 reveals that “checklist” only features in the stories reported in Appendix H. We should add it as a technique, possibly in section 6.4.2.7 “Data Checking”.&New row added to \autoref{tab:MethodsDataChecking}\\\hline
  %
  \rowcolor{gray}
  2&It’s been pointed out to me that the detailed table of “Hazop Guidewords” in appendix F of DSG v3 conflates “Hazop data considerations” with “Hazop data guidewords”. The reality is that the contents of the column “Hazop data considerations” are the HAZOP guidewords and the contents of the column “Hazop data guidewords” are the sub-property “considerations” (or meta properties) of the data property in the “Property” column.

  Taking the first row in the table as an example: The data property of “integrity” can be considered for “correctness” by asking the HAZOP questions of what the impact of loss of data, partial loss of data, incorrect data, and multiple data might have on this property (given the context and situation of the questions framing).&It's \autoref{tab:HazopShort} which was wrong -- the guidewords were labelled as considerations. I've re-titled the column header in that table. Also added text prior to the table and at the head of \autoref{bkm:guidewords}.\\\hline
\end{longtable}

\subsection{Submitted by Mike Parsons}
\begin{longtable}[H]
{|L{\dsiwgColumnWidth{0.08}}|L{\dsiwgColumnWidth{0.46}}|L{\dsiwgColumnWidth{0.46}}|}
\hline
\TableHeadColour{Num} & \TableHeadColour{Issue} & \TableHeadColour{Action}\\
\hline
  \endfirsthead
  \hline\TableHeadColour{Num} & \TableHeadColour{Issue} & \TableHeadColour{Action}\\\hline
  \endhead
  \endfoot
  \endlastfoot

1&Eurocontrol issue, for potential addition to \autoref{bkm:accidents}: \href{https://www.eurocontrol.int/sites/default/files/publication/files/summary-briefing-eurocontrol-nm-systes-outage-3-april-2018.pdf}{https://www.eurocontrol.int/sites/default/files/ publication/files/summary-briefing-eurocontrol- nm-systes-outage-3-april-2018.pdf}&**TBD** Not certain that it really fits within the Guidance - the cause is not really a data issue, although the symptoms were loss of data.\\\hline
\end{longtable}

\subsection{Submitted by Rob Oates}
\begin{longtable}[H]
{|L{\dsiwgColumnWidth{0.08}}|L{\dsiwgColumnWidth{0.46}}|L{\dsiwgColumnWidth{0.46}}|}
\hline
\TableHeadColour{Num} & \TableHeadColour{Issue} & \TableHeadColour{Action}\\
\hline
  \endfirsthead
  \hline\TableHeadColour{Num} & \TableHeadColour{Issue} & \TableHeadColour{Action}\\\hline
  \endhead
  \endfoot
  \endlastfoot

1&Breast screening error, for potential addition to \autoref{bkm:accidents}: \href{https://www.bbc.co.uk/news/health-43973652}{https://www.bbc.co.uk/news/health-43973652}&**TBD**.\\\hline
\end{longtable}

\clearpage
\subsection{Submitted by Thor Myklebust}
\begin{longtable}[H]
{|L{\dsiwgColumnWidth{0.08}}|L{\dsiwgColumnWidth{0.46}}|L{\dsiwgColumnWidth{0.46}}|}
\hline
\TableHeadColour{Num} & \TableHeadColour{Issue} & \TableHeadColour{Action}\\
\hline
  \endfirsthead
  \hline\TableHeadColour{Num} & \TableHeadColour{Issue} & \TableHeadColour{Action}\\\hline
  \endhead
  \endfoot
  \endlastfoot

  1 &
  Thor's original suggestion was to include properties of ``Analysability'' and ``Testability''.
  The meeting agreed that a third property, ``Explainability'', would be useful, particularly
  for ML/AI systems.
  It was also agreed that ``Testability'' should be generalised to ``Verifiability''. &
  **TBD** MP to address Dec 2022.\\\hline
\end{longtable}

\subsection{Submitted by Oscar Slotosch}
\begin{longtable}[H]
{|L{\dsiwgColumnWidth{0.08}}|L{\dsiwgColumnWidth{0.46}}|L{\dsiwgColumnWidth{0.46}}|}
\hline
\TableHeadColour{Num} & \TableHeadColour{Issue} & \TableHeadColour{Action}\\
\hline
  \endfirsthead
  \hline\TableHeadColour{Num} & \TableHeadColour{Issue} & \TableHeadColour{Action}\\\hline
  \endhead
  \endfoot
  \endlastfoot
% TGR Start
\rowcolor{gray}
  1 &
  When Distribution was added to \autoref{tab:issues}, Oscar pointed out that
  Data Distribution faces two main challenges: 1) Integration and 2) Communication.
  Communication is always a risk to data loss, network problems, hackers.
  The current text for ``Distribution'' addresses the first.
  Mark Templeton sees this as part of a more general weakness in the document,
  that it is primarily concerned with data as an object -
  we really need to review the document to ensure that free-flowing data gets addressed.
  \dsiwgTextBF{GOOD CANDIDATE FOR 4.0} &
  Text amended.\\\hline
  %TGR End
\end{longtable}


\subsection{Submitted by Tim Rowe}
\begin{longtable}[H]
{|L{\dsiwgColumnWidth{0.08}}|L{\dsiwgColumnWidth{0.46}}|L{\dsiwgColumnWidth{0.46}}|}
\hline
\TableHeadColour{Num} & \TableHeadColour{Issue} & \TableHeadColour{Action}\\
\hline
  \endfirsthead
  \hline\TableHeadColour{Num} & \TableHeadColour{Issue} & \TableHeadColour{Action}\\\hline
  \endhead
  \endfoot
  \endlastfoot

  1 &
  Many different things are called "categories" – there are data categories, dark data categories
  and dazzle data categories, and I’m not sure “category” means the same thing in each case.
  & 
  \\\hline
  %
  %TGR Start
  \rowcolor{gray}
  2 &
  Table 13, p39ff. It is not clear whether these are mitigation “methods” or “techniques”?
  & Closed. Although they are probably better described as techniques, this could involve changes to all tables, so no change needed.  %
  \\\hline
  %TGR End
  
  3 &
  Table 13, p39ff. The mitigations identified for dark data and dazzle data don’t seem to align
  with these.
  &
  \\\hline
  %
  %TGR Start
  \rowcolor{gray}
  4 &
  Section 7, Q5: “As with the Manufacturer, the Health Organisation needs to Identify Data
  Artefacts that are potential sources of safety hazards and understand the context of their use
  in its lifecycle.” – what does “its” refer to? It appears to be the Health Organisation,
  but I can't make sense of that.
  & Text clarified
  %TGR End
  \\\hline
  %
  % TGR Start
\rowcolor{gray}
  5 &
  Section J.2. In the main list of data categories, ML training data is the “Machine Learning”
  category. Here it’s only one of three machine learning categories.
  Also, here it’s called “Training Data”, and there is already a “Staffing and Training” category.
  % TGR End
  &
  \\\hline
  %
  % TGR Start
  \rowcolor{gray}
  6 &
  Section J.2 “Test data” is not a category in the main list of data categories.
  Here it appears to correspond to the “Verification” category.
  & Text clarified
  \\\hline

\rowcolor{gray}
  7 &
  Section J.2 “Validation Data” is not a data category in the main list of data categories,
  and is here described as being used for “Verification”, not Validation.
  & Text clarified
  \\\hline
  % TGR End
  %
  8 &
  On an internal editing note, is there a reason some acronyms are sometimes included using the \textbackslash gls command and its variants and sometimes just written out in the text?
  &
  \\\hline
\end{longtable}


\clearpage
\subsection{Issues from DSIWG meeting minutes}
\begin{longtable}[H]
{|L{\dsiwgColumnWidth{0.08}}|L{\dsiwgColumnWidth{0.46}}|L{\dsiwgColumnWidth{0.46}}|}
\hline
\TableHeadColour{Num} & \TableHeadColour{Issue} & \TableHeadColour{Action}\\
\hline
  \endfirsthead
  \hline\TableHeadColour{Num} & \TableHeadColour{Issue} & \TableHeadColour{Action}\\\hline
  \endhead
  \endfoot
  \endlastfoot

  1 &
  DSIWG meeting \#62 minutes, section 3:
  There was further discussion on the issue raised by MT around the possibility of a missing HAZOP
  guide word related to “Sufficiency”.
  
  For the Hazop we may need guidewords such as “Spurious”, “Erroneous”, “Incorrect timing” or
  “Unwanted duplication”.
  
  The problem could be termed “Spuriousity” or “Spuriousness” or “Unexpected” or “Unsolicited”. In
  the end “Unexpected” is probably the best phrasing.
  
  MP suggested that there may be parallels with ‘Dark Data’ which need to be explored.
  
  Action 62.2 (MP): Consider a ‘Dark Data’ type of treatment for unwanted or unexpected data
  
  It was thought that a new guidance section on messaging issues may be required, including aspects
  such as timing, noise, masking, CRC defeat. There are really two aspects: delivery of the unexpected
  message itself and then unexpected payload within the message.
  
  Note that we can also get issues with continuous data feeds. A fundamental issue is the breaking the
  ‘contract’ between sender and receiver (as often documented in Interface Control Documents).
  
  It was also noted that some of the existing property definitions probably need a refresh.&
  Issue 3.4 of the Guidance introduced the Goldilocks property, and the new guide word ``insufficient''.

  **TBD** The Guidance remains weak on communications issues, and we should consider this a candidate for future enhancements, possibly through an appendix focussed on communication / data transfer issues. Deferred to 4.0\\\hline
  %
  \rowcolor{gray}
  2 &
  Incorporate new chapter on dazzle / distracting / disruptive data. &
  Done -- incorporated as an appendix. Needed to go immediately after Dark Data, which altered the appendix numbering following this point.\\\hline
  %
  3 &
  DSIWG \#67 minutes, section 2: Review presentation on sensor data by Carl Jackson, for potential inclusion: \href{https://scsc.uk/file/gd/JM\_-\_20220126\_SCSCDataThoughts-1325.pptx}{https://scsc.uk/file/gd/JM\_- \_20220126\_SCSCDataThoughts-1325.pptx} & Ask Carl if he could write an appendix format.
  \\\hline
  %
  4 &
  DSIWG \#67 minutes, section 3: Topics for future inclusion:
    \begin{itemize}
        \item Communication of data,
        \item Data flows,
        \item Tool assurance,
        \item Free format data,
        \item Re-arranging the tables to make easier to use,
        \item Conflicting / distributed data elaboration.
    \end{itemize}
    & First two addressed above. Tool assurance - Oscar to write appendix (1-2 pages). Last 3 are deferred to 4.0.
    \\\hline
  %
  5 &
  DSIWG \#68 minutes, section 3: Consider DSTL ``Crumbs'' booklet, for possible enhancements to guidance. Also worth referencing.
  \href{https://assets.publishing.service.gov.uk/government/uploads/system/uploads/attachment\_data/file
    /1023385/20210901-Crumbs\_biscuit\_book\_DIGITAL.pdf}
       {https://assets.publishing.service.gov.uk/government/ uploads/system/uploads/attachment\_data/file
/1023385/20210901- Crumbs\_biscuit\_book\_DIGITAL.pdf}& MDT to add referrence, deferring further work to 4.0\\\hline
  %
  6 &
  Related to action 68.1: DSIWG \#68 minutes, section 5 and DSIWG \#69 minutes, section 9: Black Swan and Dragon King events. MP and PH work for possible future inclusion. DSIWG \#70 minutes, section 2 provides a possible basis for Guidance text.& Text almost ready, for new appendix. PH to prepare (with MP).\\\hline
  %
  \rowcolor{green}
  7 &
  Related to action 68.2: DSIWG \#68 minutes, section 7 and DSIWG \#69 minutes, section 10: Migrating, Porting and Importing Data. Probably for MP and MT.& MP to prepare.\\\hline
  %
  8 &
  DSIWG \#68 minutes, section 8: Consider new BSI guidance that addresses data, for possible impact on Guidance:
  \href{https://www.bsigroup.com/en-GB/standards/bsi-flex-236-v1.0-landing}{https://www.bsigroup.com/en-GB/standards/ bsi-flex-236-v1.0-landing} & Deferred to 4.0.\\\hline
  %
  9 &
  Related to action 69.2: DSIWG \#69 minutes, section 3: Consider whether more guidance should be added about lifecycles, and the possible alignment of data lifecycles with software development lifecycles.& Needs an expert in Agile. Deferred to 4.0.\\\hline
  %
  10 &
  Related to action 69.3: DSIWG \#69 minutes, section 3: Incorporate PMck diagram regarding data lifecycles alignment with other lifecycles.& Diagram not yet received.\\\hline
  %
  11 &
  Action 69.4: DSIWG \#69 minutes, section 4: Machine Learning section of Guidance needs elaboration and update. Initially assigned to CJ.& Need to contact Carl - action MDT or MP.\\\hline
  %
  12 &
  Related to action 69.4: DSIWG \#69 minutes, section 5: Incorporate note on Data aggregation. Development of note initially assigned to MA. & Note not yet received. Defer to 4.0.\\\hline
  %
  13 &
  DSIWG \#69 minutes, section 6: New accident for possible inclusion: Golden Ray Car Carrier Accident:
  \href{https://youtu.be/z3b4Cuot4C4}{https://youtu.be/z3b4Cuot4C4}
  & MDT to review and write up.\\\hline
  %
  14 &
  DSIWG \#70 minutes, section 2: New accident for possible inclusion (if not addressed in Black Swan text): Pudding Lane Data from the Great Fire of London:
  \href{https://en.wikipedia.org/wiki/Great\_Fire\_of\_London}{https://en.wikipedia.org/wiki/Great\_Fire\_of\_London}
  & WIll be addressed in Black Swan text.\\\hline
  %
  15 &
  DSIWG \#70 minutes, section 3: New accident for possible inclusion:
  \href{https://www.hsmsearch.com/fatality-mobile-elevating-work-platform}
       {https://www.hsmsearch.com/ fatality-mobile-elevating-work-platform}
  & MDT to review and write up\\\hline
  %
  16 &
  DSIWG \#70 minutes, section 3: New accident for possible inclusion:
  \href{https://www.flightglobal.com/safety/klm-737-used-whole-runway-for-take-off-after-intersection-data-slip-up/148742.article?utm\_source=pocket\_mylist}
       {https://www.flightglobal.com/safety/ klm-737-used-whole-runway-for-take-off-after- intersection-data-slip-up/ 148742.article?utm\_source=pocket\_mylist}
  & MDT to review and write up\\\hline
  %
  17 &
  DSIWG \#70 minutes, section 3: New accident for possible inclusion:
  \href{https://www.theregister.com/2022/07/08/capstone\_software\_bug/}{https://www.theregister.com/2022/07/08/ capstone\_software\_bug/}
  & MDT to review and write up\\\hline
  %
  18 &
  DSIWG \#70 minutes, section 3: New accident for possible inclusion:
  \href{https://www.theregister.com/2020/07/29/esa\_soho\_space\_extenders/}
       {https://www.theregister.com/2020/07/29/ esa\_soho\_space\_extenders/}& MDT to review and write up\\\hline
  %
  19 &
  DSIWG \#70 minutes, section 3: New accident for possible inclusion:
  \href{https://www.bbc.co.uk/news/uk-61829661}{https://www.bbc.co.uk/news/uk-61829661}
  & MDT to review and write up\\\hline
  %
  \rowcolor{green}
  20 &
  Action 71.1: DSIWG \#71 minutes, section 2: Add Homophones/Homonyms explicitly to the guidance. Initially assigned to MP.& MP to write up.\\\hline
  %
  \rowcolor{green}
  21 &
  Related to action 71.1: DSIWG \#71 minutes, section 2: If Homophones/Homonyms are not added to the guidance, incorporate What3Words issue as an accident. Links:
  \begin{itemize}
  \item \href{https://www.bbc.co.uk/news/technology-56901363}{https://www.bbc.co.uk/news/ technology-56901363}
  \item \href{https://cybergibbons.com/security-2/why-what3words-is-not-suitable-for-safety-critical-applications/}
    {https://cybergibbons.com/security-2/ why-what3words-is-not-suitable-for-safety- critical-applications/}
  \item \href{https://techcrunch.com/2021/04/30/what3words-legal-threat-whatfreewords/}
    {https://techcrunch.com/2021/04/30/
      what3words-legal-threat-whatfreewords/}
  \end{itemize}
  & Add the 2 (or more) real accidents that occurred. See latest draft minutes. MDT to address.

  MDT: The three linked documents do not reference specific events, so insufficient to write up as an accident.
  However \href{https://www.telegraph.co.uk/news/2021/06/01/rescuers-directed-china-australia-what3words-app-regional-accent/}{https://www.telegraph.co.uk/news/2021/06/01/ rescuers-directed-china-australia-what3words-app- regional-accent/} does give more concrete examples.\\\hline
  %
  22 &
  Action 71.2: DSIWG \#71 minutes, sections 4 and 15. Also DSIWG \#72 minutes, section 5 and DSIWG \#73 DRAFT minutes, section 5 : Consider additional data risk types (Abandoned/Derelict/Magic
Numbers, etc.) in the list of types and in the articles& Newsletter article initially. Consider for Guidance by 4.0.\\\hline
  %
  23 &
  Action 71.3 and 71.4: DSIWG \#71 minutes, section 5: Consider the addition of Security Properties. Initially assigned to PH/DA/RO
  & Unclear whether this is appropriate for Guidance. Defer to 4.0.\\\hline
  %
  24 &
  Action 71.5: DSIWG \#71 minutes, section 6: Following presentation of work on Data life Cycle by AM,
  \begin{itemize}
  \item Establish if any of his work can be published within the DSIWG and
  \item Consider a structuring similar to that used in security standards or ISO26262
  \end{itemize}
    & Defer to 4.0\\\hline
  %
  % TGR Start
  \rowcolor{gray}
  25 &
  DSIWG \#71 minutes, section 7: New accident for possible inclusion:
  NHS Covid pass down, leaving some passengers struggling to board flights:
  \href{https://www.bbc.co.uk/news/uk-62599489}{https://www.bbc.co.uk/news/uk-62599489}
  & Agreed to exclude.\\\hline
  %
  \rowcolor{gray}
  26 &
  DSIWG \#71 minutes, section 7: New accident for possible inclusion:
  Self-Driving Vehicles – Data is key
  \href{https://eandt.theiet.org/content/articles/2022/09/can-the-road-network-cope-with-self-driving-vehicles}
       {https://eandt.theiet.org/content/articles/2022/09/ can-the-road-network-cope-with-self-driving-vehicles}
  & Agreed to exclude.\\\hline
  %
  \rowcolor{gray}
  27 &
  DSIWG \#71 minutes, section 9: New accident for possible inclusion:
  Fly-by-wire "data" problem (see the second paragraph):
  \href{https://qr.ae/pNZM0P}{https://qr.ae/pNZM0P}
  & MDT: Rejected. Article is an answer to a question.
  The incident mentioned is apocriphal,
  with no specific date or reference,
  but is the kind of thing that can happen
  if checks are not carried out.\\\hline
  %
  \rowcolor{gray}
  28 &
  DSIWG \#71 minutes, section 9: New accident for possible inclusion:
  Experienced crew struggled with instrument flight after 737 lost autopilots:
  \href{https://www.flightglobal.com/safety/experienced-crew-struggled-with-instrument-flight-after-737-lost-autopilots/140072.article}
       {https://www.flightglobal.com/safety/ experienced-crew-struggled-with-instrument-flight- after-737-lost-autopilots/140072.article}
       & MDT: Rejected. Autopilot systems are for workload reduction,
       and should never be flight critical. The article does not
       indicate any equipment failures other than the autopilots
       and implies that the issue was pilot error.
       There is no indication of any specific data failings.
       Therefore out of scope for the Guidance.\\\hline
% TGR End
  %
  29 &
  DSIWG \#71 minutes, section 9: Article to consider for impact on Dark Data appendix:
  \href{https://www.pinterest.co.uk/pin/756464068674050138/}{https://www.pinterest.co.uk/pin/ 756464068674050138/}
  & MP to review example for inclusion.\\\hline
  %
  30 &
  Related to action 72.1: DSIWG \#72 minutes, section 2: Consider Tom Tom's suggestions for the Guidance: A-Structure items were:
  \begin{enumerate}
  \item Introduction: improve to describe what the standard is meant for (history, adaption of other standards, key issues, achievement goal)
    {\color{red} [Missing history, key issue of data safety, how it looks when safety achieved]}
  \item Scope: improve description (to what and where it applies and where it doesn't apply) {\color{red} [Exclusions, tighter description, rationale (esp. for management), benefits to programmes]}
  \item Tailoring/references with other standards
  \item Objectives: don't combine it with outputs (they should end up in a list of work products) {\color{red} [Need to have a more obvious safety argument structure supported by evidence (work products – all WP need unique refs.); need better cross-references]}
  \item Input list/section [What is needed to begin, what are expected outcomes]
    {\color{red} [Note: current guidance missing various distinctions of types of dynamic data, e.g. transmitted vs. stored data.]}
  \item Requirements for compliance (objectives are formulated like this already, the DSAL list with "notes" could be reformulated to cover this approach) 
  \item Implementation guidance (for requirements, like description with examples)
  \item List/section of work products (evidence of achievement)
  \end{enumerate}
  & A-Structure text added to previous column after meeting on 22 December 2022. Address any that can be fitted in. Majority may have to wait to 4.0.\\\hline
  %
  31 &
  DSIWG \#72 minutes, section 6: RR suggestions from other standards documents.
  & Part of Tom Tom issue. Deferred to 4.0\\\hline
  %
  32 &
  DSIWG \#72 minutes, section 6: Possibly include the new work that MP and PH are doing an ontology for data risks. May not yet be ready (Dec 2022).
  & Newsletter first, same as above. Maybe 4.0.\\\hline
  %
  % TGR Start
  \rowcolor{gray}
  33 &
  DSIWG \#72 minutes, section 7: New accident for possible inclusion:
  UK government breaches of email security:
  \href{https://www.theguardian.com/politics/2022/oct/31/braverman-admits-personal-email-work-six-times-apology-secret}
       {https://www.theguardian.com/politics/2022/oct/31/ braverman-admits-personal-email-work-six-times- apology-secret}
  & Agreed to exclude.\\\hline
  % TGR End
  %
       34 &
       DSIWG \#72 minutes, section 7: New accident for possible inclusion:
       UK government handling of migrants in processing centres:
       \href{https://www.bbc.co.uk/news/uk-63477371}{https://www.bbc.co.uk/news/uk-63477371}
  & Agreed to exclude.\\\hline
  %
  35 &
  Indexing and editorial enhancements& Tim Rowe to look into this.\\\hline
  %
  36 &
  DSIWG \#73 DRAFT minutes, section 2: Consider Tom Tom's lower priority issues, from the B-Considerations list:
  \begin{enumerate}
    \item Conceptual model of content (elements of Data Safety in relation to others, i.e. properties and categories)
    \item Process overview diagram (according to Risk Management ISO) 
    \item Referencing of requirements to chapters1
    \item Recommended metrics (part of measurement)
      {\color{red} [Need to be able to justify effort e.g. with metrics.]}
    \item Diagrams from Appendix into the main part
      {\color{red} [Need map of how DSG fits with typical lifecycles: V, Agile, ‘Data Processing Pipeline’]}
    \item (Data Safety in the context of V-Model)2
    \item Regional differences
      {\color{red} [Round the world; also maturity of organisational capability in this regard. Can we use a CMM-type model?]}
    \item Maturity model of Data Safety
    \item Life cycle of data safety
    \item Decision trees (BPMN) for process
    \item Recommended analysis methodologies to achieve Data Safety (i.e. STPA)
      {\color{red} [Need higher-level system hazards to feed down to data level. Also need to be able to propose higher-level hazards based on understanding of lower-level data.]}
    \item Verification and validation
    \item Appendices should be chapters (with introduction, generic, specific)
    \item {\color{red} Also need to add communications between different functions: data safety / security / system level. Need to mention processes that support comms.}
    \item {\color{red} Also need to enable size/estimation/costing of data safety work (via work products, size of system, etc.)}
  \end{enumerate}
  & \\\hline
  %
  37 &
  Action 73.1: DSIWG \#73 DRAFT minutes, section 2: Consider production of a short note which could be used as an appendix to the guidance on lessons learnt using the guidance at TomTom. Action assigned to JK and RR.
  &\\\hline
  %
  38 &
  Action 73.2: DSIWG \#73 DRAFT minutes, section 2: Consider how the guidance fits with different lifecycles considering ‘V’, Continuous Service, Agile and ‘Data Pipeline’. Assigned to MDT.
  &\\\hline
  %
  39 &
  &\\\hline
  %
  40 &
  &\\\hline
  %
\end{longtable}
