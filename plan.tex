%================================================================================
%       Safety Critical Systems Club - Data Safety Initiative Working Group
%================================================================================
%                       DDDD    SSSS  IIIII  W   W   GGGG
%                       D   D  S        I    W   W  G   
%                       D   D   SSS     I    W W W  G  GG
%                       D   D      S    I    WW WW  G   G
%                       DDDD   SSSS   IIIII  W   W   GGG
%================================================================================
%               Data Safety Guidance Document - LaTeX Source File
%================================================================================
%
% Description:
%   Data Safety Management Plan section.
%
%================================================================================
\section{\ecapitalisewords{\glsentrylong{dsmp}} (Informative)} \label{bkm:plan}

\dsiwgSectionQuote{Things get done only if the data we gather can inform and inspire those in a position to make a difference.}{Mike Schmoker}

This section gives a suggested \gls{dsmp} table of contents. It is expected that this will be needed only for aspects not already covered in a \gls{smp}, or similar. It can be merged with an \gls{smp}, if appropriate. However it may be useful to consider the distinct data perspective by using a \gls{dsmp} as well as an \gls{smp}. Regardless, a close connection should be maintained between the \gls{smp} and the \gls{dsmp}.

\Gls{dsmp} suggested contents:
\begin{enumerate}
  \item Introduction:
  \begin{itemize}
    \item Scope and Context (Sets the scene, describes the project, scenario, concept of operations, etc.);
    \item Boundaries and Interfaces (Describes the main interfaces and exchanges, with a scope boundary diagram.);
    \item
      Derived requirements (System-level requirements which impact the data safety process);
    \item \index{Data!Owner}Owners (Who owns the data under consideration as it progresses through the system?);
    \item Producers / Consumers (Who are the producers and consumers of the data the system inputs and outputs?);
    \item Assumptions;
    \item References; and
    \item Abbreviations and Acronyms.
  \end{itemize}
  \item Analysis of Assigned \gls{dsal}\index{Assurance Level!Data} and \gls{odr} Level (Implications of the data analyses.):
  \begin{itemize}
    \item \Gls{sil}, etc., Implications (What impact does the \gls{dsal}\index{Assurance Level!Data} have on the required \gls{sil}, or similar measure, and vice versa?);
    \item Development Implications (Are there any special development considerations? Derived from the \gls{sil} if there is one, otherwise what is deemed necessary for this system.);
    \item \Gls{verification} implications (derived from the \gls{sil} if there is one, otherwise what is deemed necessary for this system.);
    \item Assurance implications (derived from the \gls{sil} if there is one, otherwise what is deemed necessary for this system.); and
    \item Process / procedure implications (derived from the \gls{sil} if there is one, otherwise what is deemed necessary for this system.).
  \end{itemize}
\item
  Categories
  of Safety Data in Scope (A list of all the categories\index{Category!Data} to be considered in the system context.).
  \item Data Requirements Analysis:
  \begin{itemize}
    \item Lifecycles\index{Lifecycle!Data} (What data lifecycles are to be used?);
    \item Specific Targets (Are there any qualitative or quantitative targets for the data?); and
    \item Security Considerations (How will security be managed in this context? Are there any security / safety conflicts? Are there any security-related causes of data \glspl{hazard}?).
  \end{itemize}
  \item Management Approach (How will the organization manage the data safety risks?):
  \begin{itemize}
    \item Organization;
    \item Responsibilities;
    \item Authorisations; and
    \item Approvals and Signoffs.
  \end{itemize}
  \item Justification Approach (How will the safe usage of the data be justified, e.g., as part of the Safety Case Report?). 
  \item Analyses / \glspl{verification} to be Performed (What analyses or checks are to be performed on the data?).
  \item Documents to be Produced (The list of documents to be produced related to data aspects.).
  \item Appendix: \gls{dsal}\index{Assurance Level!Data} Guidelines Response (Tailored version of the tables from this document. What is considered applicable / useful and what is not?).
\end{enumerate}
