%-*-LaTeX-*-=====================================================================
%       Safety Critical Systems Club - Data Safety Initiative Working Group
%================================================================================
%                       DDDD    SSSS  IIIII  W   W   GGGG
%                       D   D  S        I    W   W  G   
%                       D   D   SSS     I    W W W  G  GG
%                       D   D      S    I    WW WW  G   G
%                       DDDD   SSSS   IIIII  W   W   GGG
%================================================================================
%               Data Safety Guidance Document - LaTeX Source File
%================================================================================
%
% Description:
%   This file contains all of the front matter for the guidance, i.e. before the
%   Table of Contents and main body of the text.
%
%================================================================================

%
%A blank line is important here, to get the right vertical space
%

\pagestyle{FirstPageFrontCover}
%
%

%ISBN-13: 9798373548373\todo{Update to new ISBN}

%ISBN-10: 1981662464

\acrshort{scsc} Publication Number: SCSC-127I%\todo{Guessed 3.6 should be SCSC-127I. Check before release.}

This work is licensed under the Creative Commons Attribution 4.0 International License. To view a copy of this license, visit \href{http://creativecommons.org/licenses/by/4.0/}{http://creativecommons.org/licenses/by/4.0/} or send a letter to Creative Commons, PO Box 1866, Mountain View, CA 94042, USA\@. You are free to share the material in any form and adapt the material for any purpose providing you attribute the material to the \gls{scsc} Data Safety Initiative Working Group, reference the source material, include the licence details above, and indicate if any changes were made.  See the license for full details.

This document was prepared using the \LaTeXe\/ typesetting system.

Editing and typesetting by Mark Templeton, \cbstart supported by Tim Rowe\cbend.

Cover design by Paul Hampton.

\glsreset{scsc} %So that SCSC is spelt out in full on the next page as well!


\clearpage
\pagestyle{ContinuationInsideFrontCover}

The \gls{scsc} is the professional network for sharing knowledge regarding safety-critical systems. \cbstart It brings together: 
\begin{itemize}
	\item engineers and specialists from a range of disciplines working on safety-critical systems in a wide variety of industries;
	\item academics researching the arena of safety-critical systems;
	\item providers of the tools and services that are needed to develop the systems; and
	\item the regulators who oversee safety.
\end{itemize}
\cbend
Through publications, seminars, workshops, tutorials, a web site and, most importantly, at the annual \gls{sss}, it provides opportunities for these people to network and benefit from each other’s experience in working hard at the accidents that don’t happen. It focuses on current and emerging practices in safety engineering, software engineering, and product and process safety standards.

This document was written by the \gls{dsiwg}, which is convened under the auspices of the \gls{scsc}. The document supports the \gls{dsiwg}'s vision, which is to have clear guidance that reflects emerging best practice on how data (as distinct from software and hardware) should be managed in a safety-related context. This update takes account of the consensus that a process-based guidance document will complement existing safety management processes, making it more usable. It was formally released at
\gls{sss}'24, 13--15 February 2024,
details of which may be found at
\href{https://scsc.uk/e1007}{https://scsc.uk/e1007}%\todo{Update e797 reference}.

Comments on this document are actively encouraged. These can be emailed to: 
\begin{center}
  \href{mailto:comments@data-safety.scsc.uk}{comments@data-safety.scsc.uk}.
\end{center}

Alternatively, a comments submission form is available at:
\begin{center}
  \href{http://scsc.uk/data-comments.html}{data-safety.scsc.uk/comments}.
\end{center}

While the authors and the publishers have used reasonable endeavours to ensure that the information and guidance given in this work is correct, all parties must rely on their own skill and judgement when making use of this work and obtain professional or specialist advice before taking, or refraining from, any action on the basis of the content of this work. Neither the authors nor the publishers make any representations or warranties of any kind, express or implied, about the completeness, accuracy, reliability, suitability or availability with respect to such information and guidance for any purpose, and they will not be liable for any loss or damage including without limitation, indirect or consequential loss or damage, or any loss or damage whatsoever (including as a result of negligence) arising out of, or in connection with, the use of this work. The views and opinions expressed in this publication are those of the authors and do not necessarily reflect those of their employers, the \gls{scsc} or other organizations.

\thispagestyle{empty}% Suppress page number on this intro page
\clearpage
\cbstart % KEEP: Probably want to keep this \cbstart...\cbend pair, as \maketitle always updated
\maketitle
\cbend % KEEP
\thispagestyle{empty}% Suppress page number on this title page

\clearpage
\section*{Change History}
\thispagestyle{empty}% Suppress page number on Change History page

\addtocounter{table}{-1}% Number this table as zero, so the first one in ChangeLog table is 1
  
\begin{longtable}
  {%
    |L{\dsiwgColumnWidth{0.10}}|L{\dsiwgColumnWidth{0.20}}|L{\dsiwgColumnWidth{0.55}}|L{\dsiwgColumnWidth{0.15}}|%
  }%
  \hline
  \TableHeadColour{Version} & \TableHeadColour{By} & \TableHeadColour{Status} & \TableHeadColour{Date}\\
  \hline
  \endfirsthead
  \hline
  \TableHeadColour{Version} & \TableHeadColour{By} & \TableHeadColour{Status} & \TableHeadColour{Date}\\
  \endhead
  \hline
  \endfoot
  \endlastfoot
  {1.0} & The \acrshort{dsiwg} Team & {First draft for external review} & {31-JAN-2014}
  \\\hline
  {1.1} & {The \acrshort{dsiwg} Team} & {(Internal edition for \acrshort{dsiwg} use only)} & {09-DEC-2014}
  \\\hline
  {1.2} & {The \acrshort{dsiwg} Team} & {For publication at \acrshort{sss}'15} & {23-JAN-2015}
  \\\hline
  {1.3} & {The \acrshort{dsiwg} Team} & {For publication at \acrshort{sss}'16} & {29-JAN-2016}
  \\\hline
  {2.0} & {The \acrshort{dsiwg} Team} & {For publication at \acrshort{sss}'17} & {30-JAN-2017}
  \\\hline
  {3.0} & {The \acrshort{dsiwg} Team} & {For publication at \acrshort{sss}'18} & {26-JAN-2018}
  \\\hline
  {3.1} & {The \acrshort{dsiwg} Team} & {For publication at \acrshort{sss}'19} & {01-FEB-2019}
  \\\hline
  {3.2} & {The \acrshort{dsiwg} Team} & {For publication at \acrshort{sss}'20} & {11-FEB-2020}
  \\\hline
  {3.3} & {The \acrshort{dsiwg} Team} & {For publication at \acrshort{sss}'21} & {09-FEB-2021}
  \\\hline
  {3.4} & {The \acrshort{dsiwg} Team} & {For publication at \acrshort{sss}'22} & {08-FEB-2022}
  \\\hline
  {3.5} & {The \acrshort{dsiwg} Team} & {For publication at \acrshort{sss}'23} & {07-FEB-2023}
  \\\hline
  {3.6} & {The \acrshort{dsiwg} Team} & {For publication at \acrshort{sss}'24} & { 13-FEB-2024}
  \\\hline
\end{longtable}
%
\subsection*{Changes Since the Last Edition}%\todo{Update this whole section}
In this update to the Guidance, one update has been made that introduces a slight incompatibility
to previous updates of the 3.x series.
Within \autoref{bkm:guidance}, item DD.10 has been deleted from \autoref{tab:MethodsDataDesign}
and inserted as DI.13 into \autoref{tab:MethodsDataProcedures}.
This concerns the technique ``Update Comparison'', that when data has been updated,
it can be compared with its previous value.
The change has been made because \autoref{tab:MethodsDataProcedures} is concerned with the
verification of data, and is therefore a more appropriate location for this technique.

Other updates incorporated within this version of the guidance consist primarily of:%\todo{Update this later}
\begin{itemize}
\item \dsiwgRef{Section}{bkm:dataVerification} has been renamed from
  ``Data Implementation'' to ``Data Verification'', as it is a more accurate title.
\item \autoref{tab:MethodsDataProcedures} has been renamed from
  ``Mitigation methods: data implementation'' to ``Mitigation methods: data verification''
  as the new title is more descriptive of the techniques within the table.
  \item Collection of accidents enhanced:
    \begin{itemize}
     \item New \autoref{tab:IncidentsByDomain}: A listing of all the accidents within that
       appendix by domain, such as Air or Maritime.
     \item New accident in \dsiwgRef{Appendix}{bkm:incacc:horizon}: Failure of the Post Office Horizon accounting system.
     \item New accident in \dsiwgRef{Appendix}{bkm:incacc:agincourt}: Failure to understand capabilities of English longbow led to devestating losses.
     \end{itemize}
  \item \dsiwgRef{Appendix}{bkm:MachineLearning} enhanced by the addition of new
    \autoref{bkm:MachineLearning:Hallucinations} to address the problem of hallucinations.
%  \item New \dsiwgRef{Appendix}{bkm:autonomy} presents issues and concerns that should be considered when employing AI or autonomy. 
   \item Modified \dsiwgRef{Appendix}{bkm:darkdata} by merging this appendix on Dark Data with the previous appendix on Dazzle Data, and adding \autoref{tab:DarkDazzleComparison}, which compares the properties of these two kinds of data.
%   \item New \dsiwgRef{Appendix}{bkm:radish} to introduce the Radish tool,
%  which is being developed to assist in the application of this Guidance document.

  \item A number of small adjustments were also made to the text,
  where further clarity had been recommended by users of the document. 
%        including increasing the visibility and accuracy of hyperlinks within the electronic versions of the document.
\end{itemize}

Since the last edition, \Gls{ai} systems, particularly those based on
\Glspl{llm}\index{Language Model, Large}\index{LLM|see{Language Model, Large}} have received a
lot of attention.
These are very heavily data-driven systems, but most of the hazards that have been identified are
at the societal, system, or algorithmic level and so not within the scope of this guidance.
However, issues of  \hyperref[tab:issues]{biasing, interpretation and, arguably, falsification}
may arise, and those sections have been reviewed to ensure they remain applicable to this rapidly
changing field.
Similarly, the \hyperref[bkm:guidance:dataproperties]{Data Properties}
Completeness\index{Completeness!Property}, Analysability\index{Analysability Property},
and Explainability\index{Explainability Property} have been reviewed to ensure their continued
applicability.

To assist users of earlier 3.x versions of the guidance in ensuring that their existing data safety arguments have not been impacted by this update, a version of this document is available which has been annotated with change bars. The annotated version is available at \href{http://scsc.uk/scsc-127I}{http://scsc.uk/scsc-127I}


\subsection*{Future work}
%An \gls{scsc} Ontology Working Group is currently developing a formal ontology for risk management.\todo{Update this later}
%Once this model is sufficiently mature, the intention is to apply it to data safety risk management and thus formalise the terminology and conceptual relationships used in this guidance.
%An introduction to the ontological modelling was provided in the proceedings to \acrshort{sss}’20 \cite{citation:Banham2020} and a progress update on the work was presented at \acrshort{sss}’22.

MCA Ltd has worked with the \gls{dsiwg} to develop a prototype software tool to assist in the automation of the processes described in this guidance document.
A working version of the tool has been developed and organizations that could benefit from the use and further development of the tool are urged to contact MCA via the \gls{dsiwg}.
%A description of the tool is presented in \dsiwgRef{Appendix}{bkm:radish}.

In addition to improvements to the guidance resulting from the work of these two sub-groups,
a number of improvements to the guidance are currently planned.
These improvements are intended to clarify the application of the data safety process
and include:
\begin{itemize}
\item the addition of a process flow diagram,
\item further detail on the assurance of communications and data flows,
\item data safety considerations associated with distributed data sets and Blockchain,
\item addition of new treatments to the tables in \autoref{bkm:guidance},
\item review of the tables of treatments, with the aim of making them easier to use,
\item further explanation of some treatments, where their use or benefit is not immediately apparent, 
\item reordering of parts of the document to improve readability, especially as regards likelihood, 
\item further detail on tool assurance,
\item harmonisation of language and guidance on how organizations may expand the tables to incorporate their own internal processes.
\item guidance on the application of the Data Safety Culture Questionnaire,
\end{itemize}
Several of these changes are likely to
cause parts of the document to be re-ordered --- they have therefore been deferred to the next major update, in version 4.0 of the guidance.

If you or your organization are interested in learning more about the work of the \gls{dsiwg} or joining either of the sub-groups,
please visit the \gls{scsc} website, where more information including contact details may be found on the ``Working groups'' section of the site. For direct access to the \gls{dsiwg} area of the site, please visit \href{https://scsc.uk/gd}{https://scsc.uk/gd}
%
%
%
\subsection*{Related working groups}
The \gls{scsc} sponsors initiatives to develop methods and techniques through a number of working groups. These groups each address safety aspects peculiar to their domain, including data aspects when appropriate. The current list of working groups includes:
%
\begin{itemize}
  \item Assurance Cases,
  \item Autonomous Systems Safety,
  \item Security Aspects of Safety,
  \item Safe AI,
  \item Safer Complex Systems. 
\end{itemize}
%
The newest addition is the \acrfull{saiwg}, which aims to capture cross-domain best practice and guidance on key topics within the design, evaluation, assurance, and approval of safety systems that use or are developed using \acrfull{ai}, bringing together emerging standards and key results from the incredible amount of research being conducted into AI safety.

The working group was launched at SSS’24. It will conduct regular meetings, workshops, and publications to share knowledge and experience on various topics related to AI and safety systems, such as co-ordination of safety with other disciplines, evaluation of risk, and mapping of terminology and language.

A current list of working groups, with links to further details on the work of each group, with contact details of each lead engineer may be found here: \href{https://scsc.uk/g}{https://scsc.uk/g}.
%
\clearpage
%
%\thispagestyle{empty}% Suppress page number on this intro page

% Really ugly hack (during DSIWG #37) to make the previous stuff end after an even page
\makeatletter		% Horrible hack to make contents on odd page
\dsiwg@intblankpage
\makeatother

\pagestyle{FirstPageFrontMatter}
\section*{Foreword}

\dsiwgSectionQuote{Data is here. Data is growing. Data is causing harm.}{}

\dsiwgTextBF{Data is here}:
Data is becoming ever more important in our lives: influencing, managing and even controlling many critical aspects.
The use of AI systems is a new, exciting but potentially hazardous use of data. \acrfull{llm} based systems are trained on vast amount of data, and it this data which enables them to be useful.
Some of this data is related to our personal safety and well-being.
Consider, for example, the importance of data defining the layout of railway signals,
data that indicates the position of underwater obstructions in nautical channels or data that
is used to train a vision recognition system to detect tumours in medical images.
Organizations now make significant decisions (including safety-related decisions) based solely on data held in systems.
Hence, organizations need to safely manage, control and process their data.
In particular, key \index{Property!Data}Data Properties that preserve safety must be actively managed.

\dsiwgTextBF{Data is growing}: There are at least two reasons why the use of data has grown and, equally important, why it is expected to continue to grow. The  first relates to the rapid expansion of the area loosely termed ``Big Data'', including the use of large data sets to support machine learning and artificial intelligence applications. The second is the growing use of systems of systems, where data is the lifeblood that connects together disparate elements and allows a cohesive capability to be built. Put simply, the need to address data-related issues is a pressing problem and will continue to be so.

\dsiwgTextBF{Data is causing harm}: Strictly speaking,
data can neither cause nor prevent harm.
However, mistakes introduced in data, or the inappropriate use of data, within safety-related systems have been factors in a number of documented accidents and incidents. Examples include: aircraft attempting to take off from the wrong runway (and consequently crashing); ships running aground; and patients being exposed to higher than planned doses of radiation.

Against this background, the \gls{dsiwg} was established under the auspices of the \gls{scsc}. The \gls{dsiwg}'s aim is to develop clear, cross-sector guidance that reflects emerging best practice on how data (as opposed to software or hardware) should be managed in a safety-related context.
For the most part, this guidance is based on well-established techniques,
and it has been designed to be compatible with current safety standards and to integrate with existing safety management systems.
What is new, however, is the explicit and relentless focus on data, making it a ``first-class citizen'' within system safety analyses.
By doing so, this guidance should help organizations identify, analyse, evaluate and treat data-related risks, thus reducing the likelihood of data-related issues causing harm in the future.

\clearpage
\section*{Quick Start Guide}
\pagestyle{ContinuationPageFrontMatter}

\dsiwgSectionQuote{Data really powers everything that we do.}{Jeff Weiner}

The following bullets provide a single-page introduction to Data Safety Guidance. For first-time readers this should help place individual sections within an appropriate context; it should also help returning readers quickly navigate the document's contents.

\begin{itemize}
  \item Systems are changing. The role of data is becoming more prominent. Hence, data needs to be considered as a ``first-class citizen'' in system safety analyses. This will help mitigate organizational and system-level risks associated with the use of data.

  \item A Data Safety Management Process has been developed. This is based on four phases:
    \begin{itemize}
      \item Establish Context;
      \item Identify Risks;
      \item Analyse Risks; and
      \item Evaluate and Treat Risks.
    \end{itemize}
	\item The underlying principles and an overview of the process may be found in \autoref{bkm:principlesprocess}.
	\item Definitions and abbreviations (associated with normative text) are listed in \autoref{bkm:definitionsabbreviations}.
	\item The objectives associated with, and the outputs produced by, each phase are provided in \autoref{bkm:objectivesoutputs}.
	\item The activities of each phase (and associated tailoring information) are described in \autoref{bkm:activitiestailoring}.
	\item Additional guidance information for each phase is contained in \autoref{bkm:guidance}.
  \item A worked example is provided in \autoref{bkm:workedexample}.
  \item A collection of Appendices provide more detail, including:
    \begin{itemize}
      \item A discussion illustrating how the underlying principles link to the objectives (\dsiwgRef{Appendix}{bkm:principlesobjectives}); 
      \item An Organization Data Risk assessment questionnaire (\dsiwgRef{Appendix}{bkm:assessment});
      \item A Data Safety Culture questionnaire (\dsiwgRef{Appendix}{bkm:culture});
      \item A questionnaire to help assess ``data maturity'' of a supplier (\dsiwgRef{Appendix}{bkm:maturity});
      \item A list of Data Categories (\dsiwgRef{Appendix}{bkm:categories});
      \item A collection of Hazard and Operability Study Guidewords (\dsiwgRef{Appendix}{bkm:guidewords});
      \item The suggested contents of a Data Safety Management Plan (\dsiwgRef{Appendix}{bkm:plan});
      \item A summary of accidents and incidents in which data was potentially a causal factor (\dsiwgRef{Appendix}{bkm:accidents});
      \item A discussion of topics loosely related to system lifecycles (\dsiwgRef{Appendix}{bkm:lifecycle});
      \item Considerations regarding Machine Learning (\dsiwgRef{Appendix}{bkm:MachineLearning});
%      \item A discussion of the risks of AI and autonomy (\dsiwgRef{Appendix}{bkm:autonomy});
      \item An introduction to the concepts of both Dark and Dazzle Data (\dsiwgRef{Appendix}{bkm:darkdazzledata});
      \item The concepts of Black Swan, Dragon King, Perfect Storm and Pudding Lane data (\dsiwgRef{Appendix}{bkm:cygnology}).
      \item Considerations for the assurance and qualification of data-handling tools (\dsiwgRef{Appendix}{bkm:tools}).
%      \item An introduction to the Radish tool, that has been developed to assist in the application of the guidance within this document (\dsiwgRef{Appendix}{bkm:radish}).
      \item Issues that may arise when migrating, porting, importing or exporting data (\dsiwgRef{Appendix}{bkm:migration}).            
      \item Some of the data issues that made management of the Covid-19 virus difficult (\dsiwgRef{Appendix}{bkm:Covid19});
      \item Examples of ways that \glspl{dsal} may be customised, with particular focus on likelihood (\dsiwgRef{Appendix}{bkm:DsalCustomisation});
      \item Lists of acronyms, definitions and glossary entries (\dsiwgRef{Appendix}{bkm:acronyms}); and
      \item A collection of references (\dsiwgRef{Appendix}{bkm:references}).
    \end{itemize}
\end{itemize}

%
% Table of Contents
%
%\section*{Contents}
\makeatletter		% Horrible hack to make contents on odd page
\dsiwg@intblankpage
\makeatother
\setcounter{tocdepth}{2}
\maxtocdepth{subsection}        %...and set the starting condition for TOC
\tableofcontents

%
% Following were commented out in Version 3.0
% But are included in 3.0.1 as tables and figures are now labelled.
%
%\section*{List of Tables}
%\medskip
\listoftables
%
%\section*{List of Figures}
%\medskip
\listoffigures
\cleardoublepage%ensures body of document will start on a right-hand page
