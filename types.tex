%================================================================================
%       Safety Critical Systems Club - Data Safety Initiative Working Group
%================================================================================
%                       DDDD    SSSS  IIIII  W   W   GGGG
%                       D   D  S        I    W   W  G   
%                       D   D   SSS     I    W W W  G  GG
%                       D   D      S    I    WW WW  G   G
%                       DDDD   SSSS   IIIII  W   W   GGG
%================================================================================
%               Data Safety Guidance Document - LaTeX Source File
%================================================================================
%
% Description:
%   Data Categories section.
%
%================================================================================
\section{Data Categories -- Detail (Informative)\index{Category!Data|textbf}} \label{bkm:categories}

\dsiwgSectionQuote{It's difficult to imagine the power that you're going to have when so many different sorts of data are available}{Tim Berners-Lee}

\autoref{tab:Categories} provides additional information, in the form of explanations and lists of typical containers, for the identified Data Categories.

\begin{longtable}{|C{\dsiwgColumnWidth{0.06}}|L{\dsiwgColumnWidth{0.17}}|L{\dsiwgColumnWidth{0.17}}|L{\dsiwgColumnWidth{0.38}}|L{\dsiwgColumnWidth{0.22}}|}
  \caption{Categories of safety-related data: detailed definitions}
  \label{tab:Categories}
  \\\hline\TableHeadColourC{No.} & \TableHeadColour{Category} & \TableHeadColour{Description} & \TableHeadColour{Explanation} & \TableHeadColour{Typical containers}\\\hline
  \endfirsthead
  \caption[]{Categories of safety-related data: detailed definitions (continued)}
  \\\hline\TableHeadColourC{No.} & \TableHeadColour{Category} & \TableHeadColour{Description} & \TableHeadColour{Explanation} & \TableHeadColour{Typical containers}\\\hline
  \endhead
  \multicolumn{5}{r}{\sl Continued on next page}
  \endfoot\endlastfoot
  \hline
  \multicolumn{5}{|c|}{\dsiwgTextBF{Context}}\\
  \hline
  1 & Predictive\index{Predictive Data|textbf} & Data used to model or predict behaviours and performance & Data for studies, models, prototypes, initial risk assessments, etc.  This is the data produced during the initial concept phase which subsequently flows into further development phases. & Prototype results, evaluations, analyses\\
  \hline
  2 & Scope, Assumption and Context\index{Scope, Assumption and Context Data|textbf} & Data used to frame the development, operations or provide context & Restrictions, risk criteria, usage scenarios, etc.\ explaining how the system will be used and any limitations of use. & Concepts of operation, Safety Case Report Part~1 \\
  \hline
  3 & Requirements\index{Requirement Data|textbf} & Data used to specify what the system has to do & Data encompassing requirements, specifications, internal interface\index{Interface!System} or control definitions, data formats, etc. & Formal specifications, interface control documents\index{Interface!Control Document}, user requirements documents, Safety Case Report Part~1\\
  \hline
  4 & Interface\index{Interface Data|textbf} & Data used to enable interfaces between this system and other systems:  for operations, initialisation or export from the system & Data that exists to enable exchange between this system and other external systems. Covers start-of-life operations (data import or migration), end-of-life operations and ongoing operational exchange of data between systems. & Protocols, schemas, interface control documents\index{Interface!Control Document}, transition plans, Extract-Transform-Load tool specifications, cleansing and filtering rules\\
  \hline
  5 & Reference or Lookup\index{Reference or Lookup Data|textbf} & Data used across multiple systems with generic usage & Data comprising generic reference information sets used by multiple systems (i.e., not produced solely for this system). Typically updated infrequently, and not specific to this system. & Dictionaries, materials information, sector data reference sets, encyclopedias\\
  \hline
  \pagebreak[4]%4 chosen as an experiment, as 1-3 didn't work. We don't want the next line at the bottom of a page
  \multicolumn{5}{|c|}{\dsiwgTextBF{Implementation}}\\
  \hline
  6 & Design and Development\index{Design and Development Data|textbf} & Data produced during development  and implementation & Data encompassing the design and development process artefacts: everything from design models and schemas to document review records.  It also includes test documents (specification and results) but not the test data itself. & Design documents, review records, hardware, software, test scripts, code inspection reports, Safety Case Report Part~2\\
  \hline
  7 & Software\index{Software Data|textbf} & Data that is compiled (or interpreted) and executed to achieve the desired system behaviour & From some perspectives it is helpful to consider software (e.g., source code) as another category of data. & Text files, configuration management systems\\
  \hline
  8 & Verification\index{Verification Data|textbf} & Data used to test and analyse the system & Data comprising the test values and test data sets used to verify the system. It might include real data, modified real data or synthetic data. It includes data used to drive stubs, and any data files used by simulators or emulators. & Test data sets, stub data, emulator and simulator files\\
  \hline
  \multicolumn{5}{|c|}{\dsiwgTextBF{Configuration}}\\
  \hline
	9 & \Gls{ml}\index{Machine Learning Data|textbf} & Data used to train the system to enable it to learn from the characteristics of the data & Data used to train, set up or adapt the system for a particular purpose or configuration. Might be subsets of real data or synthetically produced. Might have to include or exclude corner cases. & Images for pattern recognition analysis \\
	\hline
  10 & Infrastructure\index{Infrastructure Data|textbf} & Data used to configure, tailor or instantiate the system itself & Data used to set up and configure the system for a particular installation, product configuration, or network environment. & Network configuration files, initialisation files, hardware pin settings, network addresses, passwords \\
  \hline
  11 & Behavioural\index{Behavioural Data|textbf} & Data used to change the functionality of the system & Data to enable / disable or configure functions or behaviour of the system. & XML configuration files, \gls{csv} data, schemas \\
  \hline
  12 & Adaptation\index{Adaptation Data|textbf} & Data used to configure to a particular site & Data used to tailor or calibrate a system to a particular physical site or environment, incorporating physical or environmental conditions. & Configuration files \\
  \hline
    \pagebreak[4]%4 chosen as an experiment, as 1-3 didn't work. We don't want the next line at the bottom of a page
\multicolumn{5}{|c|}{\dsiwgTextBF{Capability}}\\
  \hline
  13 &  Staffing\index{Staffing Data|textbf} & Data related to staff training, competency, certification and permits & Data which allows staff to perform a function within the wider context of the safety-related system. This might include training records, competency assessments, permits to work, etc. & Human Resources records, training certificates, card systems\\
  \hline
  \multicolumn{5}{|c|}{\dsiwgTextBF{The Built System}}\\
  \hline
  14 & Asset\index{Asset Data|textbf} & Data about the installed or deployed system and its parts, including maintenance data & Data related to location, condition and maintenance requirements of the system under consideration. This might cover hardware, software and data. & Inventory, asset and maintenance database systems\\
  \hline
  15 & Performance\index{Performance Data|textbf} & Data collected or produced about the system during trials, pre-operational phases and live operations & Data produced by and about the system during introduction to service and live service itself. Includes fault data and diagnostic data. This might be the results of various phases of introduction and might include trend analysis to look for long-term problems. & Field data, Support calls, bug reports, non-compliance reports, \gls{dracas} data\\
  \hline
  16 & Release\index{Release Data|textbf} & Data used to ensure safe operations per release instance & Explanation of particular features or limitations of a release or instance. Might include specific time-limited workarounds and caveats for a release. & Release notes, \gls{cod}, Transfer documents, Safety Case Report Part 2 or Part 3\\
  \hline
  17 & Instructional\index{Instructional Data|textbf} & Data used to warn, train or instruct users about the system & Data that explains to users the risks of the systems and gives any mitigations\index{Mitigation} that might be required to be implemented by users, e.g., by process, procedure, workarounds, limitations of use. & Manuals, \glspl{sop}, online help, training courses, Safety Case Report Part 3\\
  \hline
  18 & Evolution\index{Evolution!Data|textbf} & Data about changes after deployment & Data that covers enhancements, formal changes, workarounds, and maintenance issues. It also covers data produced by configuration management activities, such as baselines or branch data. & Change requests, modification requests, issue and version data, configuration management system outputs\\
  \hline
  19 & End of Life\index{End of Life Data|textbf} & Data about how to stop, remove, replace or dispose of the system & Data covering all activities related to taking the system out of service or mothballing / storage / dormant phases. & Transition, disposal and decommissioning plans\\
  \hline
  20 & Stored\index{Stored Data|textbf} & Data stored by the system during operations & Data stored or used within the system which has end-user meaning. It might be displayed and used within the system or might be for transfer and distribution to other systems or downstream users. It is data that has some real domain meaning. & Might be stored internally within the system (e.g., in databases or text files), or transferred into or out of the system through interfaces\index{Interface!System} (e.g., Ethernet)\\
  \hline
  21 & Dynamic\index{Dynamic Data|textbf} & Data manipulated and processed by the system during operations & Data processed, transformed or produced by the system which has end-user meaning. It might be displayed and used within the system or might be for transfer and distribution to other systems or downstream users. It is data that has some real domain meaning. & Might be manipulated within the system in data structures or transferred into or out of the system through interfaces\index{Interface!System}\\
  \hline
  22 & Twinning\index{Twinning Data|textbf} &
  Data used to create and maintain a digital counterpart of a physical object or process &
  The digital twin is an up-to-date and accurate model when supplied with accurate and up-to-date data.
  This might be a model of a physical object's properties and states,
  including position, status and motion or of a process flow.
  A digital twin also can be used for monitoring, diagnostics and prognostics
  to optimize asset performance and use.
  Intelligent maintenance system platforms can use digital twins to find the root cause of problems.
  &
  Tooling / modelling environments and bespoke software implementations\\
  \hline
  \multicolumn{5}{|c|}{\dsiwgTextBF{Compliance and Liability}}\\
  \hline
  23 & Standards and Regulatory\index{Standards and Regulatory Data|textbf} & Data that governs the approaches,  processes and procedures used to develop safety systems & Data predominantly in the form of documents that describe and dictate the activities, processes, competencies etc.\ to be used for a particular development in a particular sector. & Standards documents, guidelines, legal directives and laws\\
  \hline
  24 & Justification\index{Justification Data|textbf} & Data used to justify the safety position of the system & Data used to justify, explain and make the case for starting or continuing live operations and why they are safe enough. Often passed to external bodies (e.g., regulators, Health and Safety Executive, Independent Safety Auditors) for their review. & Safety Case Report, certification case,  regulatory documents, \gls{cots} justification file, design justification file\\
  \hline
  25 & Investigation\index{Investigation Data|textbf} & Data used to support accident or incident investigations\index{Investigation!Incident/Accident} (i.e., potential evidence) & Data collected or produced during an incident or accident investigation which might be used in investigation reports, lessons learned or prosecutions. This can be process data, trace data, site data (e.g., photographs of crash site) or might be derived (accident simulations, analyses, etc.). & Incident/accident investigation reports and supporting documents\\
  \hline
  \multicolumn{5}{|c|}{\dsiwgTextBF{Meta-Property}}\\
  \hline
  +1 & Trustworthiness\index{Trustworthiness Data|textbf} & (Meta) data which tells us how much the system can be trusted & Data which provides assurance or confidence about the other data within or about the system under consideration. This might be some of the data mentioned in the other categories, but might be different. & Data audits, data quality index measures, sign-off sheets, traceability records, model database\\
  \hline
\end{longtable}
